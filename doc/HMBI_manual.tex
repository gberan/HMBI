\documentclass[11pt,letterpaper]{article}
\usepackage{epsfig}
\usepackage{setspace}
%\usepackage{fancyvrb} % for making pretty verbatim figures
% see example here: 
% http://x.squidpower.com/2006/08/26/verbatim-code-in-a-framed-figure-box-using-latex/

% Put a header on all pages after the first
\pagestyle{myheadings}
\markright{HMBI User Manual}

% set the margins
\usepackage[letterpaper,margin=1.0in,nofoot,dvips]{geometry}

% Colored text 
\usepackage[usenames]{color} 
\newcommand{\blue}[1]{{\color{blue} #1}}
\newcommand{\red}[1]{{\color{red} #1}}
\newcommand{\green}[1]{{\color{green!40!gray} #1}}

\usepackage{multicol}

\makeatletter
\makeatother

\begin{document}

\begin{spacing}{1.0}

% don't put a header on the first page
\thispagestyle{empty}


\section*{\centering {Hybrid Many-Body Interaction QM/MM Code User Manual}}

% don't put a header on the first page
\thispagestyle{empty}

\noindent
\begin{center}
Greg Beran, Kaushik Nanda, Ali Sebetci, and Yonaton Heit.  \\
Department of Chemistry, University of California, Riverside.\\
Last modified: March 2016
\end{center}

%\hline
\tableofcontents
\vspace{5mm}
%\hline

\section{Introduction}

HMBI is great!  Write a real introduction here.




\section{Compilation}

\subsection{Compilation instructions}

First, obtain the source code from CVS.  
\begin{verbatim}
  % cvs co hmbi
\end{verbatim}

This will create a directory 'hmbi' at your current path.  This directory
is referred to as \$HMBI for the remainder of these instructions.
HMBI has serial and parallel versions.  The parallel version uses
MPI, and the execution scripts are designed to use LAM/MPI.  

To compile:
\begin{itemize}
\item Change to the \$HMBI/src directory.

\item To make the serial version ("hmbi.serial"), type:
\begin{verbatim}
  % make serial
\end{verbatim}
(or simply type {\tt make}).  
\item To make both the serial and parallel versions ("hmbi.parallel"), type:
\begin{verbatim}
  % make all
\end{verbatim}
\item To make everything and install it:
\begin{verbatim}
  % make install 
\end{verbatim}
Installing copies the binaries to the \$HMBI/bin directory and creates
a useful softlink in that directory.  
\end{itemize}

To remove all compiled files for a fresh compilation, change to the \$HMBI/src
directory and type:
\begin{verbatim}
  % make clean
\end{verbatim}

Be sure to add the directory \$HMBI/bin to your \$PATH to enable you to run it.

Finally, you should modify \$HMBI/bin/hmbi.run to set the EXE\_PATH, MPIHOME, 
and LAMHOME variables.  The script is set up to use Sun Grid Engine for
parallel jobs.  Alternatively, a "hosts.txt" file listing the name of each
host can be used.  This file can be modified for other batch systems like
PBS.  


\subsection{Software structure}

The structure of the code is as follows:

\vspace{3mm}

\begin{tabular}{ll}
main.C		&- the overall driver for the program \\
params.C	&- a class to store the HMBI job parameters \\
\\
cluster.C	&- a class defining the full cluster, handles most of the real work. \\
dimer.C		&- a class defining a dimer, which consists of 2 monomers \\
monomer.C	&- a class defining a monomer, which consists of atoms \\
atom.C		&- a class defining an atom type \\
multipole.C     &- a class to store multipole expansions. \\
polarizability.C &- a class to store polarizabilities \\

supercell.C      &- a class defining the supercell for lattice dynamics\\
quasiharmonic.C  &- a class used for the quasi-harmonic approximation\\

\\
dlf\_interface.C &- interface to the DL-FIND geometry optimization package \\
opt.C		&- a class handling geometry optimization (currently deactivated, use DL-FIND)\\
\\
vector.C	&- a vector class \\
matrix.C        &- a matrix class \\
\end{tabular}

\section{Running HMBI}

\subsection{Serial version}

First, prepare your input file (see further instructions below for information
on input file format).  Given an input file 'job.in' and a desired output file
'job.out', run the job by typing:
\begin{verbatim}
 % hmbi job.in > job.out
\end{verbatim}

\subsection{Parallel version}

After creating the input file, create a script file to submit the job
to your cluster scheduler.  On our cluster (which uses the Sun Grid
Engine), the script looks something like:
\begin{verbatim}
-------------------------------------------
#!/bin/sh

#$ -cwd
#$ -V
#$ -pe mpi 24

hmbi ice16.in > ice16.out
-------------------------------------------
\end{verbatim}

This will run the job using the SGE parallel environment "mpi" using
24 processors.  You should contact your system administrator to identify
the appropriate parallel environment on your own cluster.
The \$HMBI/bin/hmbi.run script will collect the information about how many
processors to use and pass it to the HMBI code. 


Alternatively running the command:
\begin{verbatim}
hmbi ice16.in 24 > ice16.out
\end{verbatim}
will override the number of parallel processors requested in the queue script and instead enforce the amount of processors specified after the input filename. {\color{red} BE VERY CAREFUL WITH THIS!}

\section{Input format}

The HMBI input consists of a series of sections, which can be found in 
any order. A '\$$<$keyword$>$' marks the start of each section, and '\$end' 
marks the end of each.  The necessary sections are:

\begin{enumerate}
\item {\bf \$comment :} Job title information (optional)

\item {\bf \$molecule :} The molecule specification

\item {\bf \$hmbi :} HMBI keywords

\item Section(s) for one of the following qm methods:
  \begin{itemize}
  \item {\bf \$qchem :} (Q-Chem keywords) \\ 
  \item molpro has few sections \\
    \begin{enumerate}
    \item{\bf \$molpro:} (Molpro basis set keywords) \\
    \item{\bf \$molpro\_inst} (Molpro method keywords)\\
    \end{enumerate}
    \noindent {\em Optional} molpro sections include:\\ 
    \begin{enumerate}
    \item[(c)] For the CBS limit extrapolation \\
      \begin{enumerate}
      \item {\bf \$molpro\_inst\_HF} (Molpro Hartree-Fock method keywords for the CBS limit extrapolation)\\
      \item{\bf \$molpro\_CBS:} (Molpro higher basis set keywords for the CBS limit extrapolation)\\
      \end{enumerate}
    \item[(d)] For the CCSD(T) correction:
      \begin{enumerate}
        \item {\bf \$molpro\_ccsdt\_basis} (Molpro basis used for the CCSD(T) correction)\\
        \item {\bf \$molpro\_ccsdt\_mp2} (Molpro MP2 method used for CCSD(T) correction)\\
        \item {\bf \$molpro\_ccsdt\_inst} (Molpro CCSD(T) method used for the CCSD(T) correction)\\
      \end{enumerate}
    \end{enumerate}
  \item {\bf \$qe\_species :} (Quantum Espresso atom types) \\ 
  \item {\bf \$qe\_supercell :} (Quantum Espresso supercell grid) \\ 
  \item {\bf \$kpoints :} (Quantum Espresso kpoint grid) \\ 
  \end{itemize}  
              
%\item Either {\bf \$tinker} (Tinker keywords) or {\bf \$orient}
%  (Orient keywords) or {\bf \$qchem2} (secondary Q-Chem keywords)
\item Either section(s) for one of the following mm methods:
  \begin{itemize} 
    \item {\bf \$tinker} (Tinker keywords) 
    \item AIFF section:
      \begin{enumerate}
        \item {\bf \$aiff} (Orient keywords)
        \item {\bf \$damping\_factors\_aiff}(Giving damping factors.)
        \end{enumerate}
    \item {\bf \$qchem2} (secondary Q-Chem keywords)
  \end{itemize} 

\end{enumerate}

\noindent {\em Optional} sections include:

\begin{enumerate}

\item[6.] {\bf \$unit\_cell :} Periodic boundary condition specification.\\

\item[7.] {\bf \$embedding\_charges :} Specification of embedding charges
  (only works with Tinker MM, not Orient.)\\


\item[8.] {\bf \$reciprocal\_space\_points :} Define the k-points in a    
Monkhorst-Pack scheme for lattice dynamics.\\

\item[9.] {\bf \$thermal\_parameters :} Specification the range of temperatures
which thermodynamic properties are printed when performing lattice dynamics.\\ 

\end{enumerate}

\noindent A brief description of each section follows.

\subsection{Job Title/Comments: \$comment} 

	Free format; it can include any comments desired by the user.

\subsection{Molecule Specification: \$molecule}

The format for cluster specification depends on the many-body
treatment type, because some MM types require atom connectivity
information.  The molecular input specification requires only the
minimum information required to define the system for each
type of job.  {\bf Atomic positions are always in units of Angstroms.}

Regardless of the job type, the first line of the \$molecule section
contains the overall charge and spin state of the cluster.
\begin{verbatim}
        <charge> <spin>
\end{verbatim}
The beginning of each monomer is then indicated by a ``{\tt--}'' line,
followed by the monomer specification.  For example, a neutral cluster
with an overall singlet spin state would look like:

\begin{verbatim}
        $molecule
        0 1
        --
        <monomer 1>
        --
        <monomer 2>
        --
        etc.
        $end
\end{verbatim}
\noindent The details of specifying monomers for each particular job
type are discussed below.  

{\bf Note on spin states:} As discussed below, the electronic spin
state for each monomer must be specified.  The algorithms for
determining the spin state of dimers formed from these monomers are
somewhat crude.  Two singlet monomers form a singlet dimer.  Likewise,
a singlet monomer and a doublet monomer combine to form a doublet.
But if two doublets combine, for example, one can obtain either a
singlet or a triplet.  In such cases, the code assigns it to either an
overall singlet (if there is an even number of electrons) or doublet
(if there is an odd number of electrons).

 
\subsubsection{Monomer specification for the {\em ab initio} Force Field (AIFF) many-body model}

For AIFF many-body treatments, the first line of each monomer
specification contains:
\begin{verbatim}
        <charge> <spin>  <ionization potential in a.u.> <monomer string>
\end{verbatim}
The ionization potential (in hartrees) is used in the asymptotic DFT
correction to improve the polarizability predictions.  Experimental or
theoretical ionization potentials can be used.  The results are
moderately sensitive to the particular value of the potential.

The monomer string is a unique identifier for each monomer. This identifier 
is used to determine the damping factor for each monomer pair as defined by
\$damping\_factors\_aiff  

The next lines contain the monomer geometry in standard Cartesian XYZ
format.  Each line corresponds to the position of one atom:
\begin{verbatim}
     <atomic symbol> <x> <y> <z>
\end{verbatim}
A sample AIFF-type geometry specification for the (H$_2$O)$_3$ is given below.\\
\vspace{5mm}
\hrulefill
\begin{verbatim}
        $comment
          Sample AIFF-type geometry specification for water trimer.
        $end

        $molecule
        0 1
        --
        0 1 0.4638 H2O
        O     -1.201363      0.936340      0.045523
        H     -1.875414      1.146057     -0.611433
        H     -1.109807     -0.037296      0.018825
        --
        0 1 0.4638 H2O
        O     -0.044745     -1.610729     -0.008393 
        H      0.043174     -2.205957      0.745669 
        H      0.754975     -1.048497      0.017871 
        --
        0 1 0.4638 H2O
        O      1.585266      0.661860      0.055793 
        H      2.142455      1.072502     -0.615606 
        H      0.711544      1.090261     -0.036259 
        $end
     
        $damping_factors_aiff
        H2O
        H2O H2O 1.45
        $end
\end{verbatim}
\hrulefill
\vspace{5mm}
\\
The \$damping\_factors\_aiff section is use to define the damping 
function between every pair of monomers. This damping function is used to 
avoid the “polarization catastrophe” 
at short ranges when computing self-consistent
induction energies. The first line in this section includes every 
monomer type ``monomer string''. The next lines, the
strings contain every combination of monomer pairs followed by the damping
function for that pair. 

\vspace{5mm}
\begin{verbatim}
       <monomer string 1>  <monomer string 2> <damping function>
\end{verbatim}

It is important that  the number of monomer types 
is stated in the N\_MONOMER\_TYPES keyword 
in the \$aiff section and that this number match the number
of strings in the first line of the
\$damping\_factors\_aiff. Below is an example
of a system with two different monomer types which allow for different 
damping functions for monomer pairs.\\
Note: for the AIFF generation, a set of local coordinates will be
defined for each symmetrical unique monomer using their center of mass coordinates.



\vspace{5mm}
\hrulefill
\begin{verbatim}
        $comment
          Sample AIFF-type geometry specification for the phase IV
          ammonium nitrate unit cell
        $end

        $molecule
        0 1
        --
         1 1 0.8906 NH4
         N     4.309    1.359    4.529
         H     3.473    1.359    4.004
         H     4.309    0.550    5.102
         H     5.145    1.359    4.004
         H     4.309    2.169    5.102
        --
        -1 1 0.1447 NO3
         N     1.436    1.359    2.504
         O     2.494    1.359    1.894
         O     1.436    1.359    3.770
         O     0.378    1.359    1.894
        --
        -1 1 0.1447 NO3
         N     4.309    4.079    2.438
         O     5.367    4.079    3.048
         O     4.309    4.079    1.172
         O     3.251    4.079    3.048
         1 1 0.8906 NH4
         N     1.436    4.079    0.413
         H     2.272    4.079    0.938
         H     0.600    4.079    0.938
         H     1.436    3.269   -0.160
         H     1.436    4.888   -0.160
        $end
     
        $damping_factors_aiff
        NH4 NO3
        NH4 NH4 2.50
        NO3 NO3 3.80
        NH4 NO3 1.29
        $end
\end{verbatim}
\hrulefill




%These are defined as: The first atom of
%each monomer is the origin of the local coordinate system of that
%monomer. The positive z-axis of the local coordinate system is defined
%as the axis from the first atom to the second atom. xz-plane of the
%local coordinate system is the plane of the first three atoms of the
%monomer. For diatomic monomers, a dummy third atom may be added.



\subsubsection{Monomer specification for Quantum or EE-PA many-body models}
Jobs in which a lower-level quantum mechanical method (such as Hartree-Fock)
or the electrostatically embedded pairwise-additive approximation is used
use a molecular specification utilize a nearly identical molecular specification
to the AIFF many-body jobs.  The only difference is that the ionization
potential is omitted on the charge/spin line.  For example, one water
monomer would be specified as:\\
\vspace{5mm}
\hrulefill
\begin{verbatim}
        --
        0 1
        O     -1.201363      0.936340      0.045523
        H     -1.875414      1.146057     -0.611433
        H     -1.109807     -0.037296      0.018825
\end{verbatim}
\hrulefill



\subsubsection{Monomer specification for {\tt Tinker} force fields (such as Amoeba)}
The {\tt Tinker} software package requires atomic connectivity
information in addition to the atomic coordinates.  The first line of the
specification defines the charge and spin.  Subsequent lines define the
atoms as:
\begin{verbatim}
    <counter> <atomic symbol> <x> <y> <z> <atom_type> <connectivity>
\end{verbatim}
The {\tt counter} is a simple counter for the atoms whose number
starts at 1 on each monomer.  Fragment boundaries should not disrupt
chemical bonds.  As always, the {\tt xyz} Cartesian coordinates are in
Angstroms.  The {\tt atom\_type} is the molecular mechanics atom type,
and it corresponds to the atom type in the chosen force field.  Force-field
specification is described in Section~\ref{sec-tinker}.
and {\tt connectivity} is a list of up to 6 integers denoting the
other atoms to which this atom is bonded.  A sample specification for
the water trimer is shown below.

\vspace{5mm}
\hrulefill
\begin{verbatim}
        $comment
          Sample Tinker-type geometry specification for water trimer.
          The atom number corresponds to the Amoeba force field.
        $end

        $molecule
        0 1
        --
        0 1 0.4638
        1  O     -1.201363      0.936340      0.045523  22  2  3
        2  H     -1.875414      1.146057     -0.611433  23  1
        3  H     -1.109807     -0.037296      0.018825  23  1
        --
        0 1 0.4638
        1  O     -0.044745     -1.610729     -0.008393  22  2  3 
        2  H      0.043174     -2.205957      0.745669  23  1
        3  H      0.754975     -1.048497      0.017871  23  1 
        --
        0 1 0.4638
        1  O      1.585266      0.661860      0.055793  22  2  3
        2  H      2.142455      1.072502     -0.615606  23  1
        3  H      0.711544      1.090261     -0.036259  23  1
        $end
\end{verbatim}
\hrulefill
\vspace{5mm}


\subsection{HMBI Keyword Specification: \$hmbi}
A series of keywords of the format ``{\tt <parameter> = <value>}''.
See Chapter~\ref{HMBI_keywords} for more details on the specific
keywords.

\subsection{Q-Chem Keyword Specification: \$qchem and/or \$qchem2} 
	A series of Q-Chem keywords in standard Q-Chem format. This
	section becomes the \$rem section in Q-Chem inputs.   Caution,
        not all keywords work well.  For example, frozen core orbital
        keywords are messy, since the number of frozen core orbitals
        depends on whether we are looking at a monomer or dimer.

\subsection{Molpro Keyword Specification: \$molpro and \$molpro\_inst} 

A series of Molpro keywords in standard Molpro format. The basis set 
 portion is placed into the \$molpro section and the method (Hartree-Fock/MP2) is place 
into the \$molpro\_inst section in the same format as an Molpro input file.


A sample Molpro input variable sections:

\begin{verbatim}
   $molpro
   basis={
   set,orbital
   default,avtz
   set,jkfit
   default,avtz/jkfit
   set,mp2fit
   default,avtz/mp2fit
   }
   $end

   $molpro_inst
   {df-hf,basis=jkfit;}
   {df-mp2,basis=mp2fit;}
   $end
\end{verbatim}

\subsubsection{Molpro CBS limit}

HMBI can do complete basis set limit extrapolation using Molpro for energy, forces, optimizations, and hessian calculations.
 This extrapolation used two basis sets of different $\zeta$-levels to extrapolate energy to the complete basis set (Eq~\ref{CBS-HF}-\ref{CBS-MP2}).
Typically triple and quadrapole $\zeta$-levels are used.  Set the keyword {\bf CBS} to true to do a CBS limit calculation.
\begin{eqnarray}
\label{CBS-HF}
E_{HF}^{\infty} = E_{HF}^{large}+ \frac{E_{HF}^{Y}-E_{HF}^{X}}{e^{1.54(y-x)}-1} 
\end{eqnarray}
\begin{eqnarray}
\label{CBS-Coor}
E_{corr}^{\infty} = \frac{Y^3E_{corr}^{Y}-X^3E_{corr}^{X}}{Y^3-X^3}
\end{eqnarray}
\begin{eqnarray}
\label{CBS-MP2}
E_{MP2}^{\infty} = E_{HF}^{\infty} + E_{corr}^{\infty} 
\end{eqnarray}

\noindent
$X$ = small basis $\zeta$-level \\
\noindent
$Y$ = large basis $\zeta$-level\\

\noindent
The lower basis set keywords are placed into the \$molpro section. The higher basis set keywords are placed to \$molpro\_CBS section. 
The Hartree-Fock and MP2 keywords are separated into \$molpro\_inst\_HF and \$molpro\_inst respectively. All sections are 
in the same format as an Molpro input file. Note that the $\zeta$-level of 
the two basis sets must match the {\bf CBS\_BASIS1} (smaller basis set) and the
{\bf CBS\_BASIS2} (larger basis set) keywords in the \$HMBI section.

\begin{verbatim}
   $molpro
   basis={
   set,orbital
   default,avtz
   set,jkfit
   default,avtz/jkfit
   set,mp2fit
   default,avtz/mp2fit
   }
   $end

   $molpro_CBS
   basis={
   set,orbital
   default,avqz
   set,jkfit
   default,avqz/jkfit
   set,mp2fit
   default,avqz/mp2fit
   }
   $end

   $molpro_inst_HF
   {df-hf,basis=jkfit;}
   $end

   $molpro_inst
   {df-mp2,basis=mp2fit;}
   $end
\end{verbatim}

\subsubsection{Molpro CCSD(T) Correction}

HMBI can perform the CCSD(T) correction using Molpro for energy, forces, and optimizations calculations.
While Hessian calculations using the CCSD(T) correction is implemented, numeral Hessians needed to computed using numerical 
gradients in Molpro2012. This Hessians have been found to be too computationally noisy for two-body Hessians. This
correction takes the energy difference of MP2 and CCSD(T) in a small basis set (typically double $\zeta$) and 
adds it to the energy in a large basis set. (EQ~\ref{CCSD(T)-MP2} and~\ref{CCSD(T)}). 

Set the keyword {\bf CCSDT\_CORRECTION} in the \$HMBI section to true in order to do this correction. 
\begin{eqnarray}
\label{CCSD(T)-MP2}
\Delta^{CCSD(T)} = E_{CCSD(T)}^{small} + E_{MP2}^{small} 
\end{eqnarray}
\begin{eqnarray}
\label{CCSD(T)}
E_{CCSD(T)}^{large} \approx E_{MP2}^{large} + \Delta^{CCSD(T)}
\end{eqnarray}

In addition to the molpro sections or the sections used for the CBS molpro sections, three additional sections are used 
to perform the CCSD(T) correction. The keywords for the basis used for the CCSD(T) correction are
placed into the \$molpro\_ccsdt\_basis section. The keywords for the MP2 method are placed into \$molpro\_ccsdt\_mp2 section.
The keywords for the CCSD(T) method are placed into the \$molpro\_ccsdt\_inst.

A sample input for a Molpro CCSD(T) correction is below. These do not include the sections necessary for a molpro job or a 
CBS limit extrapolation molpro job. While the Molpro and CBS Molpro example
 sections used density-fitting, 
Molpro2012 does not have
density-fitting is not available for CCSD(T).

\begin{verbatim}
   $molpro_ccsdt_basis
   basis=avdz
   $end

   $molpro_ccsdt_mp2
   hf
   mp2
   $end

   $molpro_ccsdt_inst
   ccsd(t)
   $end
\end{verbatim}






\subsection{Tinker Keyword Specification: \$tinker \label{sec-tinker}}
   	A series of Tinker keywords in standard Tinker format.  This
	section becomes the *.key file for Tinker.  If this section is
        present, there should not be a \$aiff section.

\subsection{AIFF Keyword Specification: \$aiff}
   	A series of keywords to be used in the CamCasp/Dalton and Orient
        program packages.  If this section is
        present, there should not be a \$tinker section.

\vspace{3mm}
\noindent
\textbf{Keyword: N\_MONOMER\_TYPES} \\
Values:  positive integers\\
Default: 1\\
Notes:  Number of monomer types. \\

%\vspace{3mm}
%\noindent
%\textbf{Keyword: CamCaspHome} \\
%Values:  The home directory of the CamCasp program package. \\
%Default: None \\
%Example: /home/software/camcasp-5.2.00   \\

\vspace{3mm}
\noindent
\textbf{Keyword: OrientBasisSet} \\
Values:  Sadlej, cc-pV$X$Z, or aug-cc-pV$X$Z (where $X$ = D, T, or Q). \\
Default: Sadlej \\
Notes: This is the basis set used to compute the AIFF multipole moments
and polarizabilities. \\

%\vspace{3mm}
%\noindent
%\textbf{Keyword: DampingFactor} \\
%Values: Floating point number \\
%Default: None \\
%Notes: This prevents the ``polarization catastrophe'' at short ranges
%when computing self-consistent induction energies.  Typical values
%range roughly 1.5-2.0. For water, 1.45 appears to work well.  This
%parameter should be determined empirically by benchmarking against a
%set of fully quantum mechanical calculations. \\

\noindent
A sample AIFF input variable section:
\begin{verbatim}
        $aiff
        OrientBasisSet = sadlej
        N_MONOMER_TYPES = 1
        $end
\end{verbatim}

\subsection{Periodic Boundary Conditions Specification: \$unit\_cell}
	There are two options for specifying the unit cell.

	The default uses the lengths of each side of the unit cell (a,
        b, \& c) and the angles between them (alpha, beta, and gamma),
        where alpha is the angle between axes b \& c, beta is the
        angle between a \& c, and gamma is the angle between a \& b.
        Axis lengths are in Angstroms, and angles in degrees.  For
        example, here is the specification for formamide crystal,
        which has (a,b,c) = (3.5432, 8.9512, 6.9741), alpha = gamma =
        90 degrees, and beta = 101.051 degrees.

\begin{verbatim}
	$unit_cell
	3.5432 8.9512 6.9741
	90.0 101.051 90.0
	$end
\end{verbatim}
	The other format, which is activated by setting the \$hmbi
        keyword {\tt READ\_LATTICE\_VECTORS = TRUE}, looks for a list
        of three lattice vectors (as row vectors) that define the unit
        cell. This format does not have
          force and optimization implemented
        For example, if the three vectors are (7.569, 0, 0),
        (0, 5.366, 0), and (-4.745, 0, 8.552), you would input:

\begin{verbatim}
	$unit_cell
	 7.569 0.000 0.000
	 0.000 5.366 0.000
	-4.756 0.000 8.552
	$end
\end{verbatim}


\subsection{\$embedding\_charges}
	Charges are listed, in a.u., by fragment, in the same order as
        in the \$molecule section, with "--" separating each fragment.
        For example, in a system with two water molecules, you might
        have:

\begin{verbatim}
	$embedding_charges
	--
	-0.7781
	0.3891
	0.3891
	--
	-0.7781
	0.3891
	0.3891
	$end
\end{verbatim}

\subsection{Quantum Espresso}

Quantum Espresso (QE) employs periodic boundary conditions during its calculations which makes it incompatible with the energy breakdown HMBI typically employs. While it is still possible to use QE in HMBI, the typical computational savings you would see from running only monomer and dimer jobs in parallel will be absent as you will always be running the full crystal in each calculation.


\subsubsection{Quantum Espresso optimizations}
It is recommended that most optimizations be performed using the native QE optimizer (since it will be a little faster). Below is an example input file for an optimization performed in Quantum Espresso:

\begin{verbatim}
&CONTROL
  calculation = 'vc-relax',
  restart_mode = 'from_scratch',
  prefix = 'fullQE',
  disk_io = 'none',
  verbosity = 'high',
  etot_conv_thr = 2.0D-6,
  forc_conv_thr = 6.0D-4,
  outdir="./",
  nstep = 500,
/
&SYSTEM
  ibrav = 0,
  nat = 56,
  ntyp = 3,
  ecutwfc = 50.000000,
  ecutrho = 500.000000,
  vdw_corr = 'XDM',
  xdm_a1 = 0.6512,
  xdm_a2 = 1.4633,
/
&ELECTRONS
  electron_maxstep = 1500,
  conv_thr = 1.D-8,
  scf_must_converge = .TRUE.,
  mixing_beta = 0.5D0,
/
&IONS
/
&CELL
press = 0.D0,
/
ATOMIC_SPECIES
 H 1.00800 H.b86bpbe.UPF
 C 12.0100 C.b86bpbe.UPF
 O 16.0000 O.b86bpbe.UPF

ATOMIC_POSITIONS crystal
O     0.530162    0.840920    0.186651
H     0.459108    0.900130    0.258761
O     0.846839    0.492406    0.357912
H     0.882573    0.428621    0.486211
C     0.513499    0.710360    0.560818
H     0.425686    0.766789    0.610428
C     0.565131    0.606111    0.711257
H     0.517185    0.582054    0.879677
C     0.676984    0.531267    0.650798
H     0.715686    0.449865    0.770535
C     0.736955    0.562029    0.432573
C     0.687374    0.666318    0.279014
H     0.734430    0.688188    0.108769
C     0.575891    0.740155    0.344294
O     0.469838    0.159080    0.686651
H     0.540892    0.099870    0.758761
O     0.153161    0.507594    0.857912
H     0.117427    0.571379    0.986211
C     0.486501    0.289640    1.060818
H     0.574314    0.233211    1.110428
C     0.434869    0.393889    1.211257
H     0.482815    0.417946    1.379676
C     0.323016    0.468733    1.150798
H     0.284314    0.550134    1.270535
C     0.263045    0.437971    0.932573
C     0.312626    0.333681    0.779014
H     0.265570    0.311812    0.608769
C     0.424109    0.259845    0.844294
O     0.969838    0.340920    0.686651
H     1.040892    0.400130    0.758761
O     0.653161   -0.007594    0.857912
H     0.617427   -0.071379    0.986211
C     0.986501    0.210360    1.060818
H     1.074314    0.266789    1.110428
C     0.934869    0.106111    1.211257
H     0.982815    0.082054    1.379676
C     0.823016    0.031267    1.150798
H     0.784314   -0.050134    1.270535
C     0.763045    0.062029    0.932573
C     0.812626    0.166318    0.779014
H     0.765570    0.188188    0.608769
C     0.924109    0.240155    0.844294
O     0.030162    0.659080    0.186651
H    -0.040892    0.599870    0.258761
O     0.346839    1.007594    0.357912
H     0.382573    1.071379    0.486211
C     0.013499    0.789640    0.560818
H    -0.074314    0.733211    0.610428
C     0.065131    0.893889    0.711257
H     0.017185    0.917946    0.879677
C     0.176984    0.968733    0.650798
H     0.215686    1.050134    0.770535
C     0.236955    0.937971    0.432573
C     0.187374    0.833681    0.279014
H     0.234430    0.811812    0.108769
C     0.075891    0.759845    0.344294

K_POINTS automatic
1 1 3 1 1 1

CELL_PARAMETERS angstrom
10.372435976000   0.000000000000   0.000000000000
-0.000000000001   9.324338554000   0.000000000000
-0.000000000001   -0.000000000001   5.582421438000
\end{verbatim}

For troubleshooting problems with the QE optimizations please refer to the following Quantum Espresso websites: 

\begin{verbatim}
https://www.quantum-espresso.org/
https://www.quantum-espresso.org/Doc/INPUT_PW.html
\end{verbatim}

If you would prefer to use the HMBI optimizer then below is an example of how to perform the same optimization in HMBI.

\begin{verbatim}
$comment
Resorcinol form Alpha (RESORA03) optimized using b86bpbeXDM
$end

$hmbi
jobtype = opt
neglect_many_body = true
qm_path = qe
mm_path = amoeba
qm_code = qe
mm_code = qchem
periodic = false
full_qm_only = true
space_symmetry = true
lattice_symmetry = true
qe_basis = 50
qe_basis_mult = 10
disp_type = xdm-b86bpbe
disp_correction = true
max_opt_cycles = 500
pressure = 0
$end

$molecule
0 1
--
0 1
O       5.499071    7.841023    1.041966
H       4.762065    8.393120    1.444515
O       8.783780    4.591362    1.998015
H       9.154428    3.996607    2.714232
C       5.326232    6.623639    3.130721
H       4.415402    7.149804    3.407667
C       5.861780    5.651584    3.970536
H       5.364468    5.427265    4.910725
C       7.021970    4.953713    3.633028
H       7.423407    4.194698    4.301451
C       7.644017    5.240549    2.414806
C       7.129745    6.212979    1.557574
H       7.617828    6.416897    0.607196
C       5.973388    6.901455    1.921996
--
0 1
O       4.873364    1.483316    3.833177
H       5.610371    0.931219    4.235726
O       1.588656    4.732976    4.789226
H       1.218008    5.327732    5.505443
C       5.046204    2.700700    5.921931
H       5.957034    2.174535    6.198878
C       4.510656    3.672755    6.761747
H       5.007968    3.897074    7.701935
C       3.350466    4.370626    6.424239
H       2.949029    5.129640    7.092662
C       2.728419    4.083789    5.206016
C       3.242691    3.111359    4.348784
H       2.754608    2.907441    3.398406
C       4.399048    2.422883    4.713207
--
0 1
O      10.059582    3.178854    3.833177
H      10.796589    3.730950    4.235726
O       6.774874   -0.070807    4.789226
H       6.404226   -0.665562    5.505443
C      10.232422    1.961470    5.921931
H      11.143252    2.487634    6.198878
C       9.696874    0.989415    6.761747
H      10.194186    0.765096    7.701935
C       8.536684    0.291544    6.424239
H       8.135247   -0.467471    7.092662
C       7.914637    0.578380    5.206016
C       8.428909    1.550810    4.348784
H       7.940826    1.754728    3.398406
C       9.585266    2.239286    4.713207
--
0 1
O       0.312854    6.145485    1.041966
H      -0.424153    5.593388    1.444515
O       3.597562    9.395146    1.998015
H       3.968210    9.989901    2.714232
C       0.140014    7.362869    3.130721
H      -0.770816    6.836704    3.407667
C       0.675562    8.334924    3.970536
H       0.178250    8.559243    4.910725
C       1.835752    9.032795    3.633028
H       2.237189    9.791809    4.301451
C       2.457799    8.745959    2.414806
C       1.943527    7.773528    1.557574
H       2.431610    7.569610    0.607196
C       0.787170    7.085052    1.921996
$end

$unit_cell
10.372435976 9.3243385540000006 5.5824214379999999
90.0 90.0 90.0
$end

$qe_species
 H 1.00800 H.b86bpbe.UPF
 C 12.0100 C.b86bpbe.UPF
 O 16.0000 O.b86bpbe.UPF
$end

$kpoints
1 1 3 1 1 1
$end
\end{verbatim}


The following commands in the \$hmbi section are essential for using a Quantum Espresso job:

\begin{verbatim}
qm_code = qe
neglect_many_body = true
periodic = false
full_qm_only = true
\end{verbatim}

Optional commands that you may find useful:

\begin{verbatim}
qe_basis = 50
qe_basis_mult = 10
disp_type = xdm-b86bpbe
disp_correction = true
\end{verbatim}

Note: Pay extra attention to which type of dispersion you are applying. If the dispersion type is XDM then you must also specify which exchange functional you are using otherwise HMBI will assume it is PBE.

Other sections you must initialize:

\begin{verbatim}
$qe_species
 H 1.00800 H.b86bpbe.UPF
 C 12.0100 C.b86bpbe.UPF
 O 16.0000 O.b86bpbe.UPF
$end

$kpoints
1 1 3 1 1 1
$end
\end{verbatim}

The \$qe\_species section must contain the pseudopotential filename for each atom type you wish to use. The \$kpoints section contains the kpoint grid to be used.


\subsubsection{Quantum Espresso frequency calculations}
Currently the only established method for calculating frequencies when using QE in HMBi is to use Phonopy. The use of this external program ensures that the dispersion correction you employ in the optimizations will also be used in the calculations of the frequencies. To calculate frequencies using QE/Phonopy set the following in the \$hmbi section: 

\begin{verbatim}
jobtype = freq
qm_code = qe
neglect_many_body = true
periodic = false
full_qm_only = true
\end{verbatim}

HMBI will automatically generate the files necessary for running a Phonopy calculation and run these files. Since Phonopy generates the frequencies via finite difference this can take some time to calculate. Be sure to also set the \$qe\_species and \$kpoints section! If you wish to employ lattice dynamics then also set the following section:

\begin{verbatim}
$qe_supercell
2 2 2               
$end
\end{verbatim}

This will generate a supercell of dimension 2x2x2.

\subsection{Sample Input File}
Here is a sample input file on a water trimer.

\begin{verbatim}
$comment
Water Trimer
$end

$molecule
0 1
--
0 1
1  O       -0.571270    2.322980   -0.003174  22  2  3
2  H       -0.541626    2.994645    0.657473  23  1
3  H        0.319714    1.986524   -0.080413  23  1
--
0 1
1  O        2.038411    1.261129   -0.102281  22  2  3
2  H        2.545757    1.390343   -0.886606  23  1
3  H        2.008500    0.316696    0.040400  23  1
--
0 1
1  O        1.831197   -1.536089    0.156570  22  2  3
2  H        2.127985   -1.971116    0.938635  23  1
3  H        0.922213   -1.803155    0.031130  23  1
$end

$hmbi
jobtype = energy
path_qm = rimp2 
path_mm = amoeba
mm_code = tinker
periodic = false
iprint = 0
local_2_body = true
cutoff1 = 7.1
cutoff0 = 8.1
$end

$qchem
exchange = hf
basis = aug-cc-pvdz
aux_basis = rimp2-aug-cc-pvdz
correlation = rimp2
purecart = 11111
thresh = 14
scf_convergence = 8
mem_static = 200
mem_total = 2000
symmetry = false
$end

$tinker         
# Force Field Selection
PARAMETERS        /home/software/tinker/params/amoeba.prm
# Precision         
DIGITS            8
$end            
\end{verbatim}



\section{HMBI Keywords {\label{HMBI_keywords}}}

The keywords in the \$hmbi section can be entered in any order, and with 
any capitalization or spacing.  The only requirement is that they are 
listed in the format "parameter = value".  An exception is for the 
{\tt QM\_PATH} and {\tt MM\_PATH} keywords, for which the listed paths are case 
sensitive and must adhere to standard Unix/Linux rules.

The section begins with \$hmbi, and ends with \$end.  Blank lines in 
the section are allowed, but comment lines are not. A list of keywords follows:
\\

\subsection{General}

\noindent
\textbf{Keyword:} {\tt \bf JOBTYPE} \\
Values:  {\tt SP} (or {\tt SINGLEPOINT} or {\tt ENERGY}) = Do an HMBI 
         energy calculation [default] \\
	 {\tt FORCE} = Do an HMBI gradient calculation \\
         {\tt OPT} = Optimize the geometry \\
	 {\tt HESSIAN} (or {\tt FREQ} or {\tt FREQUENCY}) = 
         Do an HMBI frequency calculation.  \\
	 {\tt EXPAND} = Take the input geometry, and scale the distance of each
	 molecule from the center of mass.  A tool for developing potential
	 energy surfaces that expand/shrink the cluster intermolecular spacings.
	 Requires the keyword {\tt EXPANSION\_FACTOR}  as well. \\
         Has not been tested with SYMMETRY for {\tt EXPANSION\_FACTOR}. \\
Notes:   The {\tt EXPAND} jobtype creates two new files, a Cartesian xyz file,
	 new\_geom.xyz, that can be visualized with Molden or other software,
	 and a new input file, which is a direct copy of the original except
	 it has the new geometry. \\ \\ 


\noindent
\textbf{Keyword: QM\_PATH} \\
Values:  Any valid Unix-type path, pointing to the QM jobs 
	 (case-sensitive). \\
Default: qm\\
Notes:   The path should be local (expressed in terms of the directory from
	 which the program is being executed). \\

\noindent
\textbf{Keyword: MM\_PATH} \\
Values:  Any valid Unix-type path, pointing to the MM jobs 
	 (case-sensitive). \\
Default: mm\\
Notes:   The path should be local (expressed in terms of the directory from
	 which the program is being executed). \\

\noindent
\textbf{Keyword:  HESSIAN\_FILES\_PATH} \\
Values:  Any valid Unix-type path, pointing to the Hessian jobs 
	 (case-sensitive). \\
Default: hessian\_path\\
Notes:   The path should be local (expressed in terms of the directory from
	 which the program is being executed). \\

\noindent
\textbf{Keyword: QM\_CODE}\\
Values: QCHEM, MOLPRO, G09, DALTON, QE, ORCA\\
Default: QCHEM\\
Notes:    Tells the software with type of QM ouput to read.\\

\noindent
\textbf{Keyword:  MM\_CODE} \\
Values:  TINKER, AIFF, QCHEM, EE-PA, ORIENT, CHELPG, HIRSHFELD, CRYSTAL09 \\
Default: AIFF\\
%	 ORIENT (not yet implemented) \\
Notes:   Tells the software which type of MM output to read. \\

\noindent
\textbf{Keyword: IPRINT or PRINT\_LEVEL} \\
Values:  0 or any positive integer \\
Default: 0 \\
Notes:   Higher values print out more output \\

\noindent
\textbf{Keyword: PERIODIC} \\
Values:  True or False  \\
Default: False \\
Notes:   If True, must specify the \$unit\_cell section. \\

\noindent
\textbf{Keyword: READ\_LATTICE\_VECTORS} \\
Values:  True or False \\
Default: False \\
Notes:   If False, the code looks for axis lengths (a,b,c) and angles
         (alpha, beta, gamma) to define unit cell.  If True, the code
         looks for three lattice vectors, v1, v2, \& v3, each of the form
         (xi,yi,zi).  See the manual section regarding the \$unit\_cell
	 section for more details. \\

\noindent
\textbf{Keyword: COUNTERPOISE} \\
Values:  True or False  \\
Default: False \\
Notes:   If True and Q-chem is used for the QM, 
         Q-Chem's Jobtype = BSSE and extracts Counterpoise
	 corrected energies.  Warning: Q-Chem's BSSE jobtype is not fully
	 compatible with all Q-Chem methods (e.g. dual-basis SCF). \\

\noindent
\textbf{Keyword: EMBEDDING\_CHARGES} \\
Values:  True or False  \\
Default: False \\
Notes:   If True, it uses embedding charges to lessen the importance of 
many-body
	 terms.  The input file section \$embedding\_charges must be used in 
	 conjunction with this keyword. \\

\noindent
\textbf{Keyword: LOCAL\_2\_BODY} \\
Values:  True of False \\
Default: False \\
Notes:   Turns on smooth local truncation of pairwise interactions.  Truncated
	 2-body terms are treated at the MM level.  Set in combination with
         CUTOFF1 and CUTOFF0. \\

\noindent
\textbf{Keyword: CUTOFF1} \\
Values:  Any positive real number \\
Default: False \\
Notes:   Cutoff (in Angstroms) below which all 2-body interactions are treated 
	 at the QM level. \\

\noindent
\textbf{Keyword: CUTOFF0} \\
Values:  Any positive real number \\
Default: CUTOFF1 + 1.0  \\
Notes:   Cutoff (in Angstroms) above which all 2-body interactions are treated 
	 at the MM level.  Between CUTOFF1 and CUTOFF0, QM 2-body interactions 
	 are smoothly damped. \\
 
\noindent 
\textbf{Keyword: NEGLECT\_MANY\_BODY} \\
Values:  True or False \\
Default: False
Notes:   Turns off MM many-body terms.  Purely for debugging or testing. \\

\noindent 
\textbf{Keyword: BUILD\_FORCE\_FIELD\_ONLY} \\
Values:  True or False \\
Default: False \\
Notes:   Turns off QM terms.  Purely for debugging or testing. \\

\noindent
\textbf{Keyword: FULL\_QM\_ONLY} \\
Values:  True or False \\
Default: False\\
Notes:   Runs this software using only the QM method on the crystal. Only implemented for Quantum Espresso\\

\noindent
\textbf{Keyword: FULL\_MM\_ONLY} \\
Values:  True or False \\
Default: False\\
Notes:   Runs this software using MM method on the crystal. Only tested for TINKER\\

\noindent
\textbf{Keyword: ANALYZE\_ONLY}\\
Value: True or False\\
Default: False\\
Notes:   All QM and MM jobs have be run. Determine final value for the job.\\


\noindent
\textbf{Keyword: CREATE\_JOBS\_ONLY }\\
Value: True or False\\
Default: False\\
Notes:   Create input jobs but do not run them.\\

\noindent
\textbf{Keyword: OLD\_DIMER\_SYMM or OLDDIMERSYMM }\\
Value: True or False\\
Default: False\\
Notes:   Use the old symmetry labels for the dimers.\\

\noindent
\textbf{Keyword: PRESSURE}\\
Value: Any real number\\
Notes:   Pressure in GPa\\
              Only printed when PRESSURE is specified\\

\noindent
\textbf{Keyword: TEMPERATURE}\\
Value: 0 or any positive real number\\
Default: 298\\
Notes: Can only perform on a Frequency or quasi-harmonic calculation.\\
            In Kelvin.         \\  

\noindent
\textbf{Keyword: CBS}\\
Value: True or False\\
Default: False\\
Notes:   Complete basis set limit calculation. Can be used for energy, force, optimization, or hessian calculations. Molpro QM only.\\
Must include Keyword CBS\_BASIS1 and CBS\_BASIS2 keywords and the \$molpro\_CBS section in input file.\\

\noindent
\textbf{Keyword: CBS\_BASIS1}\\
Value: DZ, TZ, or QZ \\
Default: TZ\\
Notes: The smaller basis set in a complete basis set limit calculation.\\



\noindent
\textbf{Keyword: CBS\_BASIS2}\\
Value: DZ, TZ, or QZ \\
Default: QZ\\
Notes: The larger basis set in a complete basis set limit calculation.\\

\noindent
\textbf{Keyword: CCSDT\_CORRECTION}\\
Value: True or False\\
Default: False\\
Notes: CCSD(T) correction for energy, force, optimization, or hessian calculations. Molpro QM only.\\
Must include \$molpro\_ccsdt\_inst and \$molpro\_ccsdt\_mp2 sections in input file.%}
\\


\noindent
\textbf{Keyword: CHANGE\_VOLUME} \\
Values:  All real number \\
Default: 0.0 \\
Notes:   Expand or contract (negative to contract) volume of the unit cell isotropically. If SPACE\_SYMMETRY is true, symmetry will be preserved.\\

\noindent
\textbf{Keyword: CHANGE\_A} \\
Values:  All real number \\
Default: 0.0 \\
Notes:  Increase or decrease lattice parameter a. If SPACE\_SYMMETRY is
              true, symmetry will be preserved.\\

\noindent
\textbf{Keyword: CHANGE\_B} \\
Values:  All real number \\
Default: 0.0 \\
Notes:  Increase or decrease lattice parameter c. If SPACE\_SYMMETRY is
              true, symmetry will be preserved.\\

\noindent
\textbf{Keyword: CHANGE\_C} \\
Values:  All real number \\
Default: 0.0 \\
Notes:  Increase or decrease lattice parameter c. If SPACE\_SYMMETRY is
              true, symmetry will be preserved.\\

\noindent
\textbf{Keyword: CHANGE\_ALPHA} \\
Values:  All real number \\
Default: 0.0 \\
Notes:  Increase or decrease lattice parameter alpha. If SPACE\_SYMMETRY is
              true, symmetry will be preserved.\\


\noindent
\textbf{Keyword: CHANGE\_BETA} \\
Values:  All real number \\
Default: 0.0 \\
Notes:  Increase or decrease lattice parameter beta. If SPACE\_SYMMETRY is
              true, symmetry will be preserved.\\

\noindent
\textbf{Keyword: CHANGE\_GAMMA} \\
Values:  All real number \\
Default: 0.0 \\
Notes:  Increase or decrease lattice parameter gamma. If SPACE\_SYMMETRY is
              true, symmetry will be preserved.\\

\noindent
\textbf{Keyword: MEMORY} \\
Values:  Any positive integer \\
Default: 200 \\
Notes:   Memory (in megawords) used for a QM job. Molpro only \\


\noindent
\textbf{Keyword: COUPLE\_GRADIENTS} \\
Values:  True or False \\
Default: False \\
Notes:   Couples the gradients of the lattice parameters and the atoms. \\
         Currently only implemented for Quantum Espresso. \\
 
\subsection{AIFF}

\noindent
\textbf{Keyword: DO\_AIFF\_ELECTROSTATICS} \\
Values:  True or False\\
Default: True\\
Notes:   Include electrostatics in long-range two-body and many-body
         calculation.\\

\noindent
\textbf{Keyword: DO\_AIFF\_INDUCTION } \\
Values:  True or False\\
Default: True\\
Notes:   Include induction in electrostatic calculation.\\
         DO\_AIFF\_ELECTROSTATICS must be set to true.\\

\noindent
\textbf{Keyword: DO\_AIFF\_2BODY\_DISPERSION } \\
Values:  True or False\\
Default: True\\
Notes:   Do long-range 2-body dispersion \\

\noindent
\textbf{Keyword: DO\_AIFF\_3BODY\_DISPERSION } \\
Values:  True or False\\
Default: True\\
Notes:   Do long-range 3-body dispersion \\

\noindent
\textbf{Keyword: EWALD\_TINFOIL\_BOUNDARY} \\
Values:  True or False\\
Default: True\\
Notes:  ???\\

\noindent
\textbf{Keyword: POLARIZATION\_CUTOFF} \\
Values:  positive real numbers\\
Default: 15.0\\
Notes:  ???\\

\noindent
\textbf{Keyword: TWO\_BODY\_DISPERSION\_CUTOFF} \\
Values: positive real numbers\\
Default: 20.0\\
Notes:  ???\\

\noindent
\textbf{Keyword: THREE\_BODY\_DISPERSION\_CUTOFF} \\
Values: positive real numbers\\
Default: 10.0\\
Notes:  ???\\

\noindent
\textbf{Keyword: CORRECT\_MP2\_DISPERSION} \\
Values: True or False\\
Default: false\\
Notes:  ???\\


\noindent
\textbf{Keyword: MAX\_POLARIZATION\_CYCLES} \\
Values:  positive integers\\
Default: 100\\
Notes:  ???\\

\noindent
\textbf{Keyword: INDUCTION\_CONVERGENCE} \\
Values:  positive integers\\
Default: 5\\
Notes:  ???\\

\noindent
\textbf{Keyword: INDUCTION\_GRADCONVERGENCE } \\
Values:  positive integers\\
Default: 5\\
Notes:  ???\\

\noindent
\textbf{Keyword: INDUCTION\_ITER\_SCALING} \\
Values:  ???\\
Default: 1.0\\
Notes:  ???\\

\noindent
\textbf{Keyword: PRECONVERGE\_UNIT\_CELL\_MOMENTS} \\
Values:  True or False\\
Default: False\\
Notes:  ???\\

\noindent
\textbf{Keyword: EWALD\_KAPPA} \\
Values:  ???\\
Default: -1\\
Notes:  ???\\

\noindent
\textbf{Keyword: EWALD\_ACCURACY} \\
Values: ???\\
Default: 15\\
Notes:  ???\\

\noindent
\textbf{Keyword: RECIP\_CUTOFFX} \\
Values: ???\\
Default: -1\\
Notes:  ???\\

\noindent
\textbf{Keyword: RECIP\_CUTOFFY} \\
Values: ???\\
Default: -1\\
Notes:  ???\\

\noindent
\textbf{Keyword: RECIP\_CUTOFFZ} \\
Values: ???\\
Default: -1\\
Notes:  ???\\

\noindent
\textbf{Keyword: DIREC\_CUTOFFX} \\
Values: ???\\
Default: -1\\
Notes:  ???\\

\noindent
\textbf{Keyword: DIREC\_CUTOFFY} \\
Values: ???\\
Default: -1\\
Notes:  ???\\

\noindent
\textbf{Keyword: DIREC\_CUTOFFZ} \\
Values: ???\\
Default: -1\\
Notes:  ???\\



\subsection{Optimization And Forces}

\noindent
\textbf{Keyword: OPTIMIZER} \\
Values: DLFIND, KNITRO, CONJUGATEGRADIENT or CG, \\
        STEEPESTDESCENT or SD or STEEPEST\-DESCENT, LBFGS or L\-BFGS \\
Default: DLFIND \\
Notes: Needed only when JOB\_TYPE \= OPT. If you wish to use only \\ 
       the local optimizer then set CG, SD, or LBFGS. Recommend using \\
       DLFIND.\\


\noindent
\textbf{Keyword: MAX\_OPT\_CYCLES} \\
Values:  Positive integer \\
Default: 150 \\
Notes:   Maximum number of cycles to use in geometry optimizations.  \\
        % Default is 100 at present. \\


\noindent
\textbf{Keyword: DO\_FREQ\_AFTER\_OPT} \\
Values: True or False\\
Default: False\\
Notes: Perform frequency calculation once optimization is complete.\\ 


\noindent
\textbf{Keyword: FREEZE\_UNITCELLPARAMS} \\
Values:  True or False\\
Default: False \\
Notes:   Freezes lattice parameters during optimizations. \\
             Optimizes unit cell nuclear coordinates only.\\

 \noindent
\textbf{Keyword: FREEZE\_ATOM\_COORDINATES} \\
Values:  True or False\\
Default: False \\
Notes:   Freezes atom coordinates during optimizations.\\
             Optimizes unit cell parameters only.\\

 \noindent
\textbf{Keyword: UNFREEZE\_LATTICE\_PARAMS\_AFTER} \\
Values:  0 or positive integers\\
Default: 0 \\
Notes:   Freezes lattice parameters until a given number of
              optimization steps.\\

 \noindent
\textbf{Keyword: FDIFF\_GRAD} \\
Values:  True or False\\
Default: False \\
Notes:   Determines gradient using finite difference. For debugging and testing only.\\ 


\noindent
\textbf{Keyword: STEPSIZE} \\
Values: Postitive doubles from 0.1 - 1.0\\
Default: 0.65\\
Notes: Needed only when JOB\_TYPE = OPT and using the local optimizer. Determines step\\ 
              length when doing line-search on a rejected step.\\

\subsection{Frequency}

\noindent
\textbf{Keyword: DEUTERATED} \\
Values: True or False\\
Default: False\\
Notes: Hydrogen atoms are given the mass of deuterium atoms when determining vibrational frequencies. \\

\noindent
\textbf{Keyword: MOLECULE} \\
Values: True or False\\
Default: False\\
Notes: Frequency of a molecule instead of for the Crystal. Necessary only for Quantum Espresso\\

\noindent
\textbf{Keyword: NO\_FREQ\_DISP or NO\_FREQ\_DISPLACEMENT} \\
Values: True or False\\
Default: False\\
Notes: Does not displpace molecule if negative frequencies exist\\

\noindent
\textbf{Keyword: ARE\_FORCES\_AVAILABLE} \\
Values: True or False\\
Default: False\\
Notes: The necessary force calculations 
were already  performed. This is often the done doing the final step of an optimization.\\
 This keyword set to true if ANALYZE\_ONLY is true.\\

\noindent
\textbf{Keyword: ARE\_HESSIANS\_AVAILABLE} \\
Values: True or False\\
Default: False\\
Notes: The necessary hessian calculations
were already  performed. This keyword set to true if ANALYZE\_ONLY is true.\\

\noindent
\textbf{Keyword: SINGLE\_FILE\_MONOMER\_HESS} \\
Values: True or False\\
Default: False\\
Notes: QM monomer Hessians calculated in single job. If false, monomer
           Hessians are determined by finite difference in multiple jobs.\\
           Molpro only. Q-chem always calculates for monomer hessian with
           a single job.\\

\noindent
\textbf{Keyword: FDIFF\_HESS} \\
Values: True or False\\
Default: False\\
Notes: Determine Frequency using finite difference. For debugging and testing only.\\
Incompatible with SPACE\_GROUP equal to true.\\

\subsection{Lattice Dynamics}

\noindent
\textbf{Keyword: SUPERCELL\_JOB} \\
Values: True or False\\
Default: False\\
Notes: Perform supercell Hessian calculations to do lattice dynamics. \\
            Must include \$reciprocal\_space\_points section in input. \\

\noindent
\textbf{Keyword: SUPERCELL\_SIZE\_A} \\
Values: positive integer\\
Default: 3\\
Notes: Length the unit cell is extended along lattice length a.\\

\noindent
\textbf{Keyword: SUPERCELL\_SIZE\_B} \\
Values: positive integer\\
Default: 3\\
Notes: Length the unit cell is extended along lattice length b.\\

\noindent
\textbf{Keyword: SUPERCELL\_SIZE\_B} \\
Values: positive integer\\
Default: 3\\
Notes: Length the unit cell is extended along lattice length c.\\

\noindent
\textbf{Keyword: SUPERCELL\_ANALYZE\_ONLY} \\
Values: True or False\\
Default: False\\
Notes: Supercell Hessian calculation has already be performed. \\

\noindent
\textbf{Keyword: THERMAL\_FOR\_MULTIPLE\_TEMPERATURES} \\
Values: True or False\\
Default: False\\
Notes: Print thermodynamic properties for a range of temperature.   \\
The range is in \$thermal\_parameters section. If this section is not included.
The range is 0 to 298 K with intervals of 1 K\\

\subsection{Quasi-Harmonic Approximation}

\noindent
\textbf{Keyword: QUASIHARMONIC} \\
Values: True or False\\
Default: False\\
Notes: Performs the quasi-harmonic approximation\\

\noindent
\textbf{Keyword: ARE\_QHA\_AVAILABLE} \\
Values: True or False\\
Default: False\\
Notes: If false, frequencies for the reference cell and the
       gr\"{u}neisen is determined. The frequencies are printed
       in the .freq file.\\
       If true, all frequencies necessary are in the .freq file.\\

\noindent
\textbf{Keyword: SAVE\_QHA} \\
Values: True or False\\
Default: True [RECOMMENDED]\\
Notes: Place the jobs for each quasi-harmonic step in a separated directory. \\

\noindent
\textbf{Keyword: NUMBER\_OF\_QUASIHARMONIC\_STEPS} \\
Values: 0, 1, 2\\
Default: 0\\
Notes: If software stops in the when determining the frequencies for the Gr\"{u}neisen parameters,
            restart at a given step. 0 to start at the reference volume. 1 to start at the plus volume.
            2 to start at the minus volume.\\ 


\noindent
\textbf{Keyword: READ\_QUASIHARMONIC\_GEOMETRIES} \\
Values: True or False\\
Default: False\\
Notes: The plus and minus structures are found in PlusFreq.in and MinusFreq.in respectively instead of 
           simply expanding and compressing the reference cell.\\

\subsection{Symmetry}

\noindent
\textbf{Keyword: SYMMETRY} \\
Values: True or False\\
Default: True\\
Notes: Allows for symmetry implementation.\\
        If false all symmetry options will be set to false.\\

\noindent
\textbf{Keyword: SPACE\_SYMMETRY} \\
Values: True or False\\
Default : True when PERIODIC is true, otherwise false. Always false when 
SYMMETRY is false.\\
Notes: Uses space group symmetry to reduce the number of
dimer calculations and degrees of freedom for optimization.\\
Not implemented for system using one atom monomer units.\\
For crystals consisting of monomers that belong to high symmetry point group, \\
this option has only been tested when the lattice angles are 90$^{\circ}$\\
See doi:10.1002/jcc.23737.\\

\noindent
\textbf{Keyword: LATTICE\_SYMMETRY}\\
Values: True or False\\
Default: False, Always true when SPACE\_SYMMETRY is true. Always false when SYMMETRY is false\\
Notes:  Fixes lattice angles that are 90 or 120 degrees.\\
        Lattice constants that are the same value are kept the same during optimization\\

\noindent
\textbf{Keyword: PRINT\_SYMMETRY\_INFO}\\
Values: True or False\\
Default: False\\
Notes: Prints the number of symmetrical monomers and dimers.\\
       Prints the symmetry operators which maps atoms 
to atoms in asymmetric unit.\\

\noindent
\textbf{Keyword: MONOMER\_SYMMETRY}\\
Values: True or False\\
Default: False\\
Notes: Reduces number of QM and TINKER monomers with space group symmetry.\\

\noindent
\textbf{Keyword:AIFF\_SYMMETRY}\\
Values: True or False\\
Default: True\\
Notes: Reduce the number of monomers used by AIFF with space group symmetry.\\

\noindent
\textbf{Keyword: SYMMETRY\_TOLERANCE}\\
Values: Positive integers\\
Default: True\\
Notes: Number of digits after the decimal of the symmetry tolerance.\\


\subsection{Quantum Espresso}

\noindent
\textbf{Keyword: DISP\_CORRECTION or DISPERSION\_CORRECTION}\\
Values: True or False\\
Default: False\\
Notes: Enables dispersion correction for the Quantum Espresso job.\\

\noindent
\textbf{Keyword: DISP\_TYPE or DISPERSION\_TYPE}\\
Values: D2, XDM\-PW86PBE, XDM\-B86BPBE, XDM\-BLYP, XDM\-REVPBE, XDM\-PBESOL, XDM\-PBE or XDM\\
Default: XDM\\
Notes: Sets appropriate parameters in the Quantum Espresso job for the given dispersion type.\\

\noindent
\textbf{Keyword: FORCE\_XC}\\
Values: Any valid exchange type from Quantum Espresso.\\
Default: none\\
Notes: Enforces the given exchange potential. Be very careful with this parameter!\\

\noindent
\textbf{Keyword: QE\_BASIS}\\
Values: Positive Integers\\
Default: none\\
Notes: Defines the kinetic energy cutoff for the Quantum Espresso jobs.\\

\noindent
\textbf{Keyword: QE\_BASIS\_MULTIPLIER or QE\_BASIS\_MULT}\\
Values: Positive Integers\\
Default: none\\
Notes: Defines the kinetic energy for charge density and potential cutoff for the Quantum Espresso jobs.\\

\section{Lattice Dynamics}
\label{Chap-Lat-Dyn}
\subsection{Introduction}

HMBI has the ability to do lattice dynamics. Lattice
dynamics allows the use to determine the vibrational modes at 
various {\bf k}-points. The phonons at a particular {\bf k}-point
is found from the diagonalization of the mass-weighted
supercell dynamical matrix,
\begin{equation}
\label{eq-dyn-mat}
\hat{D}_{\alpha,\beta}(l,l^\prime,{\bf k} ) = \frac{1}{\sqrt{M_l M_{l^\prime}}}\sum_{\kappa}\frac{\partial V}{\partial R_\alpha(0)\partial R_\beta(\kappa)} \exp{\left(-2\pi i {\bf k} \cdot \delta {\bf R}_{l,l^{\prime}}(0,\kappa) \right)}
\end{equation}
where $\frac{\partial V}{\partial R_\alpha(0)\partial R_\beta(\kappa)}$  is the elements of the supercell Hessian between the coordinate $\alpha(0)$  of atom $l$ 
inside the central unit cell and coordinate $\beta(\kappa)$ of atom $l^\prime$ outside the central unit cell. ${ \delta \bf R}_{l,l^{\prime}}(0,\kappa)$ is the distance
between these two atoms.

 An advantage of fragment-based methods 
such as HMBI over other electronic structure methods such as periodic DFT or MP2 is that lattice dynamics supercell Hessian requires little additional computational 
cost compared to the standard unit cell Hessian. All the necessary QM terms are already determined in the Hessian for the unit cell. The only additional term is an
supercell Hessian determined by the MM method.

\subsection{Usage}


In order perform lattice dynamics in HMBI your \$HMBI section must include the following keyword
 ``{\bf SUPERCELL\_JOB}  = true''. By default the size of the supercell is 3 
$\times$ 3 $\times$ 3. The size of the supercell
changed with the keywords {\bf SUPERCELL\_SIZE\_A}, {\bf SUPERCELL\_SIZE\_B}, and {\bf SUPERCELL\_SIZE\_C}.
If the supercell hessian has already been calculation, set {\bf SUPERCELL\_ANALYZE\_ONLY} to true.
In addition, the \$reciprocal\_space\_points is used to determine the Monkhorst-Pack {\bf k}-point grid.
By default  3 $\times$ 3$ \times$ 3 Monkhorst-Pack {\bf k}-point grid. \\
\\
\noindent
An example of the \$reciprocal\_space\_points:

\begin{verbatim}
   $reciprocal_space_points
   3 3 3               
   $end
\end{verbatim}

\subsection{Print Thermodynamic Properties with Temperature}

HMBI uses the vibrational phonons determined from the lattice dynamic 
calculation to determine vibrational energy, isochoric
heat capacity, and entropy (among others). The temperature in Kelvin is selected
using the {\bf TEMPERATURE} keyword in the \$HMBI section. The default 
value is 298. If the user would like to view these properties over a range of
temperatures, set the keyword {\bf THERMAL\_FOR\_MULTIPLE\_TEMPERATURE} to true 
and include a \$thermal\_parameters section. 
\\
\\
\noindent
The \$thermal\_parameters section contains one line and includes the minimum 
and maximum temperature as well as the temperature intervals in the follow format.

\begin{verbatim}
    <Min Temp>  <Max Temp> <Interval>
\end{verbatim}
Here is an example \$thermal\_parameters section with the default values.
\begin{verbatim}
   $thermal_parameters
   0 298 1
   $end 
\end{verbatim}

\section{Quasi-Harmonic Approximation}

\subsection{Introduction}

The quasi-harmonic approximate is used to determine how a particular
 vibrational mode of a crystal changes with 
volume using Gr\"{u}neisen parameters $\gamma_{i}$.
\begin{equation}
\label{eq-grun}
\gamma_{i} = \frac{\partial \omega_{i}}{\partial V}
\end{equation} 
Integrating Eq~\ref{eq-grun} gives,
\begin{equation}
\label{eq-freq}
f_{i}  = f_{i}^{ref}\left(\frac{V}{V_{ref}}\right)^{-\gamma_{i}}
\end{equation}
Using this approximation, the frequency any given unit cell can be determined which 
are used to calculate the Helmholtz vibration free energy.
\begin{equation}
 F_{vib}(T) = N_A\sum_i{\left(\frac{\hbar f_{i}}{2} 
+ k_bT \ln{\left[1-exp\left(-\frac{\hbar f_{i}}{k_bT}\right)\right]}\right)}
\end{equation} 
The Helmholtz vibration free energy is included into the overall energy.
Since the Helmholtz vibration free energy is dependent on temperature, the quasi-harmonic approximate 
can be used to determine how periodic crystals expand with temperature. The quasi-harmonic approximation
is implemented specifically for temperature depend optimizations for periodic crystals but it can be applied
to single point energy calculations. Before one can used the quasi-harmonic approximation, one must
first obtain the Gr\"{u}neisen  Parameters and  Reference Frequencies for EQ~\ref{eq-freq}. Getting the
Gr\"{u}neisen  Parameters and  Reference Frequencies is discussed in the next section.

One thing to note is that the Quasi-harmonic approximation is completely compatible with lattice dynamics and it is
recommended that lattice dynamics are used. See Chapter~\ref{Chap-Lat-Dyn}.


\subsection{Derive Gr\"{u}neisen  Parameters and  Reference Frequencies}

In order to use the quasi-harmonic approximation as defined by HMBI, three sets of 
frequencies are need. The first set are the frequencies of a reference cell. The other 
two are the frequencies of the reference 
cell expanded (referred to as the PlusFreq cell) and compressed(referred to as the MinusFreq cell)
usually by 10 $\AA^3$. There are two methods to derive Gr\"{u}neisen  parameters and reference frequencies which are 
in Sections \ref{Sec-Grun-same-time} and \ref{Sec-Grun-freq-already}. Note that when determining the Gr\"{u}neisen parameters
the what the keyword {\bf JOBTYPE} is set to does not matter.\\


\subsubsection{ Determine the structure, frequencies, and Gr\"{u}neisen  parameters at the same time.}
\label{Sec-Grun-same-time}

This easiest of the two methods to set up but the slowest to run. The optimization and frequency calculation for each
of the three structures are done one at a time.
In your \$HMBI section set the following keywords:

\begin{verbatim}
    QUASIHARMONIC = true
    READ_QUASIHARMONIC_GEOMETRIES = false
    ARE_QHA_AVAILABLE = false
\end{verbatim}

This method expands and contrasts the reference cell by 10 $\AA^3$.
Note that when determining Gr\"{u}neisen parameters 
the  keyword {\bf JOBTYPE} is set to does not matter. 

If your job is interrupted, 
you can restart it at certain points. Replace your input with the structure in ``reference.in''. 
This input is either the structure of your reference cell (what your input file optimized to) or the last step in the optimization of 
your reference cell.
 Then use the keyword {\bf NUMBER\_OF\_QUASIHARMONIC\_STEPS} to set which structure (reference, 
PlusFreq, or MinusFreq) to start with.
If your reference structure did not optimize or did not finish 
calculating the frequencies, set 
{\bf NUMBER\_OF\_QUASIHARMONIC\_STEPS} to 0. If the
frequencies for the ``reference'' cell are determined but not the frequencies for the ``PlusFreq'' cell,
 set {\bf NUMBER\_OF\_QUASIHARMONIC\_STEPS} to 1. 
If both the frequencies for the  
``reference''  and ``PlusFreq'' cells are determined but not the frequencies for
the ``MinusFreq'' cell, 
Set the keyword {\bf NUMBER\_OF\_QUASIHARMONIC\_STEPS} to 2. To know which step 
you are on, check the .freq file. If the ``ReferenceFrequencies'' string
is in the .freq file, the reference frequencies have been determined. If the ``PlusVolume'' string is in the
.freq file, the PlusFreq frequencies have been determined. If the ``MinusVolume''
string is in this file, the MinusFreq frequencies are known.

The frequencies derived from each cell is in the .freq file and must be in the same directory when utilizing the 
Gr\"{u}neisen parameters.



\subsubsection{Determine Gr\"{u}neisen  Parameters Separately from Frequencies}
\label{Sec-Grun-freq-already}

This method is quicker but has a lot of little steps that must
 be taken to implement. Here, the all structures and
frequencies are determined separately and then combined to create the .freq file. While the frequencies for
the  Gr\"{u}neisen  parameters would already be obtain for the separate structure before creating the .freq file,
the final job to create the .freq file orders and matches the frequencies base on vector mode overlap to 
the reference structures modes rather than be frequency size. 

Do the following steps.

\begin{enumerate}

  \item Optimize your structure. The optimized structure is your reference structure. Determine the frequency of this structure.
  \item Set two sets of input file from the optimized reference structure. These are your
    PlusFreq and MinusFreq structures. For two structure \$HMBI, include the keywords below. Note that the keyword {\bf CHANGE\_VOLUME}
    determines the finite difference step and as defined be is 10~\AA$^3$. Feel free to change as see fit.


\begin{verbatim}
For PlusFreq:

 FREEZE_UNITCELLPARAMS = true
 CHANGE_VOLUME = 10

For MinusFreq:

 FREEZE_UNITCELLPARAMS = true
 CHANGE_VOLUME = -10
\end{verbatim}

  \item Optimize these structures and determine the frequencies.
    If you optimize and determine the frequencies using different inputs (the keyword {\bf DO\_FREQ\_AFTER\_OPT} allows
    you to do so in the same job), remember to remove the {\bf CHANGE\_VOLUME} keyword for your frequency calculation.
  \item Set up your input file to obtain your .freq file. Use the keywords below in you \$HMBI section. {\bf Do not run yet.}
    Note that when determining the Gr\"{u}neisen parameters what {\bf JOBTYPE} is set to does not matter.

\begin{verbatim}
    QUASIHARMONIC = true
    READ_QUASIHARMONIC_GEOMETRIES = true
    ANALYZE_ONLY = true
    ARE_QHA_AVAILABLE = false
\end{verbatim}

    \item Create two additional input files named PlusFreq.in and MinusFreq.in. Place the PlusFreq geometry into PlusFreq.in
      and the MinusFreq geometry into MinusFreq geometry. Only the \$molecule and \$unit\_cell sections of PlusFreq.in and MinusFreq.in are
      read.
    \item Place the following jobs into the following directions. The QM\_Path, MM\_Path, and 
      HESSIAN\_FILES\_PATH are defined in you \$HMBI file with default values being ``qm'', ``mm'', and ``hessian\_files''
      respectively.
      \begin{enumerate}
        \item Jobs for the reference structure.
        \begin{itemize}
          \item Place the reference QM force calculations into \\
            $<$base path$>$/$<$QM\_PATH$>$/reference/ 
          \item Place the reference MM force calculations into \\
            $<$base path$>$/$<$MM\_PATH$>$/reference/   
          \item Place the reference QM Hessian calculations into \\
            $<$base path$>$/$<$HESSIAN\_FILES\_PATH$>$/qm/reference/  
          \item Place the reference MM Hessian calculation into \\
            $<$base path$>$/$<$HESSIAN\_FILES\_PATH$>$/mm/reference/  
        \end{itemize}
        \item Jobs for the PlusFreq structure.
        \begin{itemize}
          \item Place the PlusFreq QM force calculations into \\
            $<$base path$>$/$<$QM\_PATH$>$/PlusFreq/ 
          \item Place the PlusFreq MM force calculations into \\
            $<$base path$>$/$<$MM\_PATH$>$/PlusFreq/   
          \item Place the PlusFreq QM Hessian calculations into \\
            $<$base path$>$/$<$HESSIAN\_FILES\_PATH$>$/qm/PlusFreq/  
          \item Place the PlusFreq MM Hessian calculation into \\
            $<$base path$>$/$<$HESSIAN\_FILES\_PATH$>$/mm/PlusFreq/  
        \end{itemize}
      \item Jobs for the MinusFreq structure.
        \begin{itemize}
          \item Place the MinusFreq QM force calculations into \\
            $<$base path$>$/$<$QM\_PATH$>$/MinusFreq/ 
          \item Place the MinusFreq MM force calculations into \\
            $<$base path$>$/$<$MM\_PATH$>$/MinusFreq/   
          \item Place the MinusFreq QM Hessian calculations into \\
            $<$base path$>$/$<$HESSIAN\_FILES\_PATH$>$/qm/MinusFreq/  
          \item Place the MinusFreq MM Hessian calculation into \\
            $<$base path$>$/$<$HESSIAN\_FILES\_PATH$>$/mm/MinusFreq/  
        \end{itemize}
      \end{enumerate}
    \item Run job to obtain the .freq file.

\end{enumerate}

\subsubsection{Incorporating lattice relaxation into the Quasiharmonic Approximation}
\label{Sec-Grun-freq-latt-relax}
While the majority of the method remains the same as the previous section, this method will allow the lattice parameters of the plus and minus structure to relax by applying external pressure. This can be important for crystals that do not have symmetry in their lattice parameters (in particular triclinic crystals).

Do the following steps.

\begin{enumerate}

  \item Optimize your structure. The optimized structure is your reference structure. Determine the frequency of this structure.
  \item Now you need to apply external pressure to obtain the structures of your Plus and Minus structure. Ideally you should only have to run a few jobs to obtain structures that are about $\pm$ 10~\AA$^3$. Unfortunately this is a somewhat blind process as you do not know what volume you will get from each pressure until you optimize your structure. Note that the naming logic of these structures is opposite to that of the fixed volume approach (ie. To obtain the expanded-volume structure (PlusFreq) you must release the external pressure). Also note that Pressure in HMBI is in units of GPa.


\begin{verbatim}

For PlusFreq:


 PRESSURE = -1

For MinusFreq:

 PRESSURE = 1
\end{verbatim}

  \item Optimize these structures and determine the frequencies.
    Once you have optimized two structures that are around 10~\AA$^3$ from your reference volume, you need to determine the frequencies using different inputs. Remember to remove the {\bf CHANGE\_VOLUME} keyword for your frequency calculation.
  \item Using the optimized reference geometry, set up your input file to obtain your .freq file. Use the keywords below in your \$HMBI section. {\bf Do not run yet.}
    Note that when determining the Gr\"{u}neisen parameters what {\bf JOBTYPE} is set to does not matter.

\begin{verbatim}
    QUASIHARMONIC = true
    READ_QUASIHARMONIC_GEOMETRIES = true
    ANALYZE_ONLY = true
    ARE_QHA_AVAILABLE = false
\end{verbatim}


    \item Create two additional input files named PlusFreq.in and MinusFreq.in. Place the PlusFreq geometry into PlusFreq.in
      and the MinusFreq geometry into MinusFreq geometry. Also update their respective \$unit\_cell sections. Only the \$molecule and \$unit\_cell sections of PlusFreq.in and MinusFreq.in are
      read.
    \item Place the following jobs into the following directions. The QM\_Path, MM\_Path, and 
      HESSIAN\_FILES\_PATH are defined in you \$HMBI file with default values being ``qm'', ``mm'', and ``hessian\_files''
      respectively.
      \begin{enumerate}
        \item Jobs for the reference structure.
        \begin{itemize}
          \item Place the reference QM force calculations into \\
            $<$base path$>$/$<$QM\_PATH$>$/reference/ 
          \item Place the reference MM force calculations into \\
            $<$base path$>$/$<$MM\_PATH$>$/reference/   
          \item Place the reference QM Hessian calculations into \\
            $<$base path$>$/$<$HESSIAN\_FILES\_PATH$>$/qm/reference/  
          \item Place the reference MM Hessian calculation into \\
            $<$base path$>$/$<$HESSIAN\_FILES\_PATH$>$/mm/reference/  
        \end{itemize}
        \item Jobs for the PlusFreq structure.
        \begin{itemize}
          \item Place the PlusFreq QM force calculations into \\
            $<$base path$>$/$<$QM\_PATH$>$/PlusFreq/ 
          \item Place the PlusFreq MM force calculations into \\
            $<$base path$>$/$<$MM\_PATH$>$/PlusFreq/   
          \item Place the PlusFreq QM Hessian calculations into \\
            $<$base path$>$/$<$HESSIAN\_FILES\_PATH$>$/qm/PlusFreq/  
          \item Place the PlusFreq MM Hessian calculation into \\
            $<$base path$>$/$<$HESSIAN\_FILES\_PATH$>$/mm/PlusFreq/  
        \end{itemize}
      \item Jobs for the MinusFreq structure.
        \begin{itemize}
          \item Place the MinusFreq QM force calculations into \\
            $<$base path$>$/$<$QM\_PATH$>$/MinusFreq/ 
          \item Place the MinusFreq MM force calculations into \\
            $<$base path$>$/$<$MM\_PATH$>$/MinusFreq/   
          \item Place the MinusFreq QM Hessian calculations into \\
            $<$base path$>$/$<$HESSIAN\_FILES\_PATH$>$/qm/MinusFreq/  
          \item Place the MinusFreq MM Hessian calculation into \\
            $<$base path$>$/$<$HESSIAN\_FILES\_PATH$>$/mm/MinusFreq/  
        \end{itemize}
      \end{enumerate}
    \item Run job to obtain the .freq file.

\end{enumerate}


\subsection{Use Gr\"{u}neisen  parameters to Thermal Expand}

Once the .freq is create, Gr\"{u}neisen  parameters can be obtain.
The vibrational frequencies at any given volume can be obtained and the crystal can be optimized as function temperature and pressure.
Define temperature (in Kelvin) with the keyword {\bf TEMPERATURE} and optionally the pressure (in GPa) with the {\bf PRESSURE}.
The default temperature is 298 K. While Quasi-harmonic approximation was designed with temperature optimization in mind, it 
can be utilized for single point energy calculations. If {\bf JOBTYPE} is set to ``HESSIAN'', it will be
changed to ``SP''. To use the Gr\"{u}neisen  parameters, .freq file must be in the same
directory and the following keywords must be in the \$HMBI section of your input file.

\begin{verbatim}
    QUASIHARMONIC = true
    ARE_QHA_AVAILABLE = true
\end{verbatim}

If lattice dynamics was used compute frequencies, the  \$reciprocal\_space\_points section  must be included in the input 
file with the same Monkhorst-Pack {\bf k}-point grid.


\section{The "Tiered" Quasi-Harmonic Approximation}

\subsection{Introduction}

This method attempts to address some of the problems associated with using the current Quasi-Harmonic Approximation (QHA). Most importantly we try to address the computational cost of using the QHA. While previously you would run the full QHA (opt, freq, and thermal expansion) at the desired level of theory (such as CCSD(T)/CBS) this can quickly become computationally prohibitive. Instead we have found that it is acceptable to perform optimizations and frequency calculations using a cheaper theory (eg. DFT) and single-point energy correct up to your desired level of theory.  

Ultimately this method breaks up the calculation of the Gibbs Free Energy into multiple parts. From statistical thermodynamics,
the Gibbs free energy combines the electronic internal energy
$U_{el}$, the Helmholtz vibrational free energy $F_{vib}$, and a
pressure-volume ($PV$) contribution.
\begin{equation}
\label{eq-gibbs}
G(T,P) =  U_{el} + F_{vib}(T) + PV
\end{equation}
In crystals at ambient pressure, the $PV$ term contributes negligibly.

Typically the electronic internal energy contribution comes from HMBI:
\begin{equation}
\label{eq-hmbi}
U_{el}^{HMBI} = E_{1-body}^{QM} + E_{SR\ 2-body}^{QM} + E_{LR\ 2-body}^{MM} +
E_{many\ body}^{MM}
\end{equation}

The Helmholtz vibrational free energy is computed from standard
harmonic oscillator vibrational partition functions as,
\begin{equation}
\label{eq-fvib}
F_{vib}(T) = N_a \sum_i {\left(\frac{\hbar \omega_i}{2} + k_b T \ln\left[1-\exp\left(-\frac{\hbar \omega_i}{k_b T}\right)\right]\right)}
\end{equation}
where $N_{a}$ is Avogadro's number, $\hbar$ is Plank's constant,
$k_{b}$ is the Boltzmann constant, and $\omega_{i}$ is the vibrational
frequency of mode $i$.  The first term corresponds to the zero-point
vibrational contribution, while the second gives the thermal
vibrational contribution.

The main difference between the previous QHA and the new method is that each of these contributions are now broken up into a series of curves which are then combined using a MATLAB script. By doing this we allow the electronic internal energy curve to become any desired level of theory which can then be combined with the frequencies which are generated using a lower level of theory. 

The following sections will outline how to go about performing these calculations, problems to look out for, and ultimately what you should expect to see as a result of your efforts. While this currently requires a lot of supervision we hope in the future to make this method user-friendly.

\subsection{Calculating the Electronic Internal Energy Curve ($U_{el}$)}

\subsubsection{Obtaining the Reference geometry}
First you must optimize the geometry of your structure. This can be done with any level of method but we recommend optimizing using Quantum Espresso and the B86bPBE-XDM pseudopotential. Prepare the input file and optimize the geometry. This optimized structure will be referred to as the reference structure from now on.

\subsubsection{Generating the E(V) curve}
Now we need to get an idea of how the electronic internal energy changes as a function of volume. There are two ways to accomplish this 1) Isotropically expand and contract the reference unit cell volume and perform a fixed-cell optimization on each struture. or 2) Apply external pressure to the reference structure and allow the structure to optimize. We recommend using the latter method as this will allow the lattice parameters to relax naturally. If you are using Quantum Espresso for the quantum calculation then for both methods it is recommended that the optimization be carried out using the Quantum Espresso optimizer instead of HMBI's.

\textbf{Isotropic E(V) curve}

\vspace{2mm}
For any desired number of points, make a copy of your reference geometry into a separate file add the following command to your \$hmbi section:

\begin{verbatim}
FREEZE_UNITCELLPARAMS = true
CHANGE_VOLUME = 10
\end{verbatim}

Then optimize the geometries.

\textbf{Anisotropic E(V) curve}

\vspace{2mm}
While this method is a little more hit and miss than the Isotropic method it can be essential for crystals whose unit cell lacks high symmetry. To accomplish this we apply external pressure to our reference structure and optimize. For HMBI this is as simple as adding the following command to the \$hmbi section:

\begin{verbatim}
PRESSURE = 1
\end{verbatim}

Note that pressure in HMBI is in units of GPa and in Quantum Espresso pressure is in units of Kbar. With this method the HMBI Final Energy will include the pressure contribution so you will need to subtract the PV term from the HMBI energy (or if using the HMBI optimizer then simply grep for the last HMBI Electronic Energy).

Since both methods are fairly repetative scripting is particularily useful at this step. The example script below will take the reference geometry, create the necessary files, alter the pressure, and submit the files to the queue:

\begin{verbatim}
#!/bin/bash

#Pressure in Kbar for QE
#These are the pressures I want to iterate over
for i in `seq 2 2 10` `seq -2 -2 -10`
do
  echo pV${i}
  mkdir pV${i}
  cd pV${i}
  cp ../reference.in resorcinol.in
  sed -i "s/this/${i}/g" resorcinol.in
  cp ../r-mp2.sh r-pV${i}resAlpha.sh
  qsub r-pV${i}resAlpha.sh
  cd ../
done
\end{verbatim}

When performing these calculations it is ideal to have at least 10 points on your E(V) curves. You also want to ensure you have a well defined energy well in addition to points out to at least 20-30~\AA$^3$ away from your reference volume in both directions. For the calculation of the $F_{vib}(T)$ you will need two structures about $\pm$ 10~\AA$^3$ away from your reference volume. It is recommended that you continue generating points until you achieve structures within 2-3~\AA$^3$ of this target.

\subsubsection{Optional: Performing Single-point Energy Corrections}
After obtaining an E(V) plot at the cheaper theory you can then single-point energy correct up to your desired level of theory. What this involves is taking your volumes and geometries of each point along the E(V) curve, setting up a new HMBI job for the desired theory, and running the jobs with the following command set in the \$hmbi section:

\begin{verbatim}
JOBTYPE = ENERGY
\end{verbatim}

This can cause your energy well to shift since the potential energy surface has changed. In order to ensure that you have sampled the bottom of the new energy well sufficiently  it is recommended you perform the SPE corrections immediately after you finish optimizing each geometry in the cheaper method. If you find that your original set of geometries does not sufficiently describe the new energy well then continue generating structures at the cheaper level of theory until you have a good E(V) curve for your desired level of theory.

\subsubsection{Fitting the E(V) curve in MATLAB}

Once you have obtained the E(V) plot at the desired energy level you will need to place the data in a file format that MATLAB can readily read in. An example of this is below:

\begin{verbatim}
#Volume (Ang^3) Energy (kJ/mol)
832.126858      -4012152.23356034
688.318397      -4012252.07190479
564.18181       -4012327.45258067
551.820153      -4012330.03371393
540.173259      -4012330.87271897
532.795677      -4012330.56600123
528.378883      -4012329.94051682
516.739809      -4012327.12134208
500.872309      -4012318.65886698
476.592477      -4012294.49410104
466.927207      -4012280.23177907
443.653594      -4012231.99856419
431.328664      -4012196.81229346
407.377253      -4012104.67778039
\end{verbatim}

Note: MATLAB is sensitive to tabs so you'll want to ensure that only one tab or space exists between your data points. Name your file in a way that you and (hopefully) others can understand 3 months from now (Ex: e$-$el$-$b86bpbeXDM.dat).

The $E(V)$ curve will then be read into MATLAB and fitted to a Murnaghan
equation of state,
\begin{equation}
E(V) = E_0 + \frac{B_0V}{B_0^\prime} \left[ \frac{\left(V_0/V\right)^{B_0^\prime}}{B_0^\prime -1} +1 \right] - \frac{B_0 V_0}{B_0^\prime -1} 
\end{equation}
where $E_0$, $V_0$, $B_0$, and $B_0^\prime$ are the fit parameters.
$E_0$ gives the electronic energy at the minimum, $V_0$ is the molar
volume at the minimum energy, $B_0$ is the bulk modulus, and
$B_0^\prime$ is the first derivative of the bulk modulus with respect
to pressure. This fit allows the MATLAB script to both interpolate and extrapolate over a range of volumes which it will generate.

\subsection{Calculating the Helmholtz Free Energy Curve ($F_{vib}(T)$)}

This section is mostly unchanged from the previous QHA. To generate the $F_{vib}(T)$ curves you must first generate your Gr\"{u}neisen parameters. To do this, you must calculate the frequencies of your reference structure. You must also calculate the frequencies for two of the structures that were optimized for use in your electronic internal energy curve (preferably ones that are around 10~\AA$^3$ away from your reference volume). After that follow the procedure outlined in Section \ref{Sec-Grun-freq-latt-relax} to generate the Gr\"{u}neisen parameters. In particular you will be using the .freq file that is generated. Note that you only need to do this for the cheaper level of theory. Or put another way: even if the energy well has shifted you do not need to generate 3 new sets of frequencies for geometries that better-describe the new energy well (the Gr\"{u}neisen parameters should take care of this).

\subsection{Calculating the Gibbs Free Energy Curve Neglecting Pressure ($G(T)$)}

The following is the MATLAB script which will generate the Gibbs Free Energy plots while taking your E(V) plot and the .freq file you generated. There are a few lines you will need to change to get this working best for your system of interest:

\begin{verbatim}
test = fopen('resorcinolAlpha.freq'); <--- Will need to become whatever your .freq name is

e_el = importdata('e-el-b86bpbeXDM.dat'); <--- Will need to become whatever
					     you labeled your E(V) plot

v_min = Energy_fit(4)-100;  <--- If you wish to change the minimum volume
				 generated (units of Ang^3)

v_max = Energy_fit(4)+100;  <--- If you wish to change the maximum volume
				 generated (units of Ang^3)

t0=0:10:100;   <--- Make your temperature range whatever you want 
		    (units of K)
t1=125:25:425;
Temp=cat(2,t0,t1);
.OR.
for temp = 0:25:300
\end{verbatim}

Important files that will be generated include:
\vspace{3mm}

\begin{tabular}{ll}
murnaghanEqFit.txt	&- Keeps a log of the fitted Murnaghan EOS parameters \\
	&- Note that this was only fit against the E(V) curve\\
freeEnergy\*.txt	&- Keeps a record of the Gibbs Free Energy at a given volume and Temperature \\
\\
fvibEnergy\*.txt &- Keeps a record of the Helmholtz Free Energy at a given volume and Temperature. \\
final\_results.txt	&- a summary of all the important values at different temperatures \\

\end{tabular}

\vspace{3mm}
Remember this script assumes that the pressure is 0 GPa.
\vspace{3mm}

\begin{verbatim}
%Script to compute Free Energy at given Temp and Volumes
%Self-generates Grueneisen parameters and Fvib calcs
clear
test = fopen('resorcinolAlpha.freq');
%test = fopen(freq_name);
ref_found = false;
plus_found = false;
minus_found = false;
ref_freqs = {};
plus_freqs = {};
minus_freqs = {};
x_old = 0;

%constants needed for the calculation
h= 6.62607363e-34; %Js
Na= 6.0221367e23; %1/mol
kb= 1.3806488e-23; %J/K
R= kb*Na;  %J/(mol K)
c= 2.99792458e8; %m/s
count_structures = 0;
%
%Step 0: Extract frequency data from file
%
tline = fgetl(test);
while ~feof(test)
    %disp(tline);
    if contains(tline,'Frequencies')
        blah = strsplit(tline);
        ref_vol = str2double(blah(2));
        ref_found = true;
        tline = fgetl(test);
        x_old = 0;
        count_structures = count_structures + 1;
    elseif contains(tline,'PlusVolume')
        blah = strsplit(tline);
        plus_vol = str2double(blah(2));
        plus_found = true;
        tline = fgetl(test);
        x_old = 0;
        count_structures = count_structures + 1;
    elseif contains(tline,'MinusVolume')
        blah = strsplit(tline);
        minus_vol = str2double(blah(2));
        minus_found = true;
        tline = fgetl(test);
        x_old = 0;
        count_structures = count_structures + 1;
    end
    
    x = str2double(tline);
    
    if (ref_found) && (plus_found) && (minus_found)
        minus_freqs = [minus_freqs x];
    elseif (ref_found) && (plus_found) && (~minus_found)
        plus_freqs = [plus_freqs x];
    elseif (ref_found) && (~plus_found) && (~minus_found)
        ref_freqs = [ref_freqs x];        
    end
    
    if(abs(x_old-x) > 2000) && (x_old~=0)
        %fprintf('x: %f\tx_old: %f\n',x,x_old);
        count_structures = count_structures + 1;
    end
    
    if(x ~= 0)
        x_old = x;
    end
    
    tline = fgetl(test);
end

count_structures = count_structures / 3; 
fprintf('count_structures: %i\n',count_structures);

%convert from a cell array to a regular array
ref_freqs = cell2mat(ref_freqs);
plus_freqs = cell2mat(plus_freqs);
minus_freqs = cell2mat(minus_freqs);

%get rid of NaN in the array
ref_freqs(isnan(ref_freqs)) = [];
plus_freqs(isnan(plus_freqs)) = [];
minus_freqs(isnan(minus_freqs)) = [];
fclose(test);

%
%Step 1: Get electronic energy data and fit it.
%
e_el = importdata('e-el-b86bpbeXDM.dat');
Volumes = e_el.data(:,1)';
Energies = e_el.data(:,2)';
[minEnergy,pos] = min(Energies);
minVolume = Volumes(pos);

F = @(a,x) a(1) + (a(2) * x)/a(3) .* (((a(4)./x).^(a(3)))/(a(3) - 1) + 1 )
 - ((a(2) * a(4) )/(a(3) - 1));

% Set initial guess parameters and perform least squares fit on the E(V)
% data.
a0 = [minEnergy, 5.0, 8.0, minVolume];
Energy_fit = lsqcurvefit(F,a0,Volumes,Energies);
fprintf('E(V) Fit Parameters:\n');
fprintf('   A0 = %.5f\n',Energy_fit(1));
fprintf('   V0 = %.3f\n',Energy_fit(4));
fprintf('   B0 = %.3f\n',Energy_fit(2));
fprintf('   B0prime = %.3f\n\n',Energy_fit(3));
Emin = Energy_fit(1);


%Keep a log of the Fit Parameters
name = sprintf( 'murnaghanEqFit.txt');
fitEq = fopen(name,'w');
fprintf(fitEq,'E(V) Fit Parameters:\n');
fprintf(fitEq,'   A0 = %.5f\n',Energy_fit(1));
fprintf(fitEq,'   V0 = %.3f\n',Energy_fit(4));
fprintf(fitEq,'   B0 = %.3f\n',Energy_fit(2));
fprintf(fitEq,'   B0prime = %.3f\n\n',Energy_fit(3));
fclose(fitEq);

%
%Step 2: Begin generating Fvib data
%

%First calculate the grueneisen parameters
grun = -1.*((log(plus_freqs./minus_freqs))./(log(plus_vol./minus_vol)));
grun(isnan(grun)) = 0;

%Volume to extrapolate from
v_min = Energy_fit(4)-100;
v_max = Energy_fit(4)+100;
vol_array = [v_min:v_max];
results = fopen('final_results.txt','w');
fprintf(results,'#Temperature (K)     Optimal Volume (Ang^3)    Free Energy (kJ/mol)
    Entropy (kJ/mol)    Enthalpy (kJ/mol)    Heat Capacity (J/(mol K))    
Internal Energy (kJ/mol)\n');

t0=0:10:100;
t1=125:25:425;
Temp=cat(2,t0,t1);
%Loop over a set of temperatures
for temp = Temp;
%for temp = 0:25:25
  %Reset Fvib values to an empty array
  fprintf('\n\nTemp = %f\n',temp);
  fvib = {};
  hvib = {};
  svib = {};
  cv = {};
  %loop over a set of volumes
  for vol = vol_array

    %calculate new frequency
    %units: m/s * (cm/m) * (1/cm) = 1/s = Hz
    w =  ( c * 100 ) .* ref_freqs .* (( vol / ref_vol ).^( -1 .* grun ));
    %Calculate zero point energy contribution
    zpe = (sum(w) * h) / 2;
    
    %If relevant calculate other term
    if temp ~= 0
     %units: J/K * K = J
     h_vib_term2 = (h .* w) ./ ( exp( ( h .* w )./( kb * temp )) - 1);
     h_vib_term2 (~isfinite(h_vib_term2))=0;
     h_vib_term2 = sum(h_vib_term2);
     s_vib = ((h .* w) ./(temp .* ( exp( ( h .* w )./( kb * temp )) - 1))) 
- (( kb ) .* log( 1 - exp( ( -1 * h .* w )./( kb * temp ))));
     s_vib(~isfinite(s_vib))=0;
     s_vib= sum(s_vib);   
     Cv = (Na/(count_structures*kb)).*( ( ((h .* w) ./(temp .* 
( exp( ( h .* w )./( kb * temp )) - 1))).^2).* exp( (h .* w )./
( kb * temp )));
     Cv(~isfinite(Cv))=0;
     Cv = sum(Cv);     
     
    else
      entropy_enthalpy = 0.0;
      s_vib = 0;
      h_vib_term2 = 0;
      Cv = 0;
    end
    
    %Sum to create Fvib
    %units: (/mol) * (J + J) * (kJ / J) = kJ/mol
    s_vib =  Na * (s_vib) / (1000 * count_structures);
    h_vib =  Na * (zpe + h_vib_term2) / (1000 * count_structures);
    f_vib = h_vib - (temp * s_vib);
    fvib = [fvib f_vib];
    hvib = [hvib h_vib];   
    svib = [svib s_vib];
    cv = [cv Cv];

  end
  %Call to calculate the gibbs free energy
  fvib = cell2mat(fvib);
  hvib = cell2mat(hvib);
  svib = cell2mat(svib);
  cv = cell2mat(cv);
  
  %
  % Step 3: Fit Fvib(V) data for current temperature
  %
  
  fvib_spline = spline(vol_array,fvib);
  spline_fvib_func = @(v) ppval(fvib_spline,v);
  
  hvib_spline = spline(vol_array,hvib);
  spline_hvib_func = @(v) ppval(hvib_spline,v);  
  
  svib_spline = spline(vol_array,svib);
  spline_svib_func = @(v) ppval(svib_spline,v); 
  
  cv_spline = spline(vol_array,cv);
  spline_cv_func = @(v) ppval(cv_spline,v); 
  
  %
  % Step 4: Define Helmholtz free energy function by combining our E(V) fit
  % and the f_vib data.
  %

  Free_energy = @(v) F(Energy_fit,v) + spline_fvib_func(v); 
  %find the minimum of the Gibb's Free Energy well
  [Vmin, Fmin, info, output] = fminbnd(Free_energy, v_min, v_max);
  
  % Print out the optimal volume and free energy
  fprintf('Optimal Volume and Free Energy:\n%f Ang^3\t%f kJ/mol\n',Vmin,Fmin);
  Smin = spline_svib_func(Vmin);
  Hmin = spline_hvib_func(Vmin);
  CVmin = spline_cv_func(Vmin);
  fprintf('Optimal Entropy, Enthalpy and Cv:\n%f J/mol\t%f kJ/mol\t%f
 J/(mol K)\n',Smin*1000,Hmin,CVmin);
  
  % Finally add the F(V) result to the plot.
  %plot(Volumes,Free_energy(Volumes) - Fmin);
  %scatter(vol_array,fvib); 
  %hold on; % keep plot active while we add curves
  %plot(vol_array,f_vib_func(vol_array));
  %plot(vol_array,spline_fvib_func(vol_array));
  %legend('Fvib data', 'Fvib lin fit', 'Fvib spline');
  %legend('E(V) data', 'Murnaghan E(V)', 'Free Energy');
  %hold off; % done with plot
  
  %
  %Step 5: Store results in 'logical' files
  %
  
  name = sprintf( 'freeEnergy%i.txt', temp );
  fileID = fopen(name,'w');
  fprintf(fileID,'#Volume (Ang^3)   Free Energy (kJ/mol)\n');
  for i = vol_array
    fprintf(fileID,'%f %f\n',i,Free_energy(i));
    %fprintf('%f %f\n',i,Free_energy(i));
  end

  fclose(fileID);
  
  name = sprintf( 'fvibEnergy%i.txt', temp );
  fileID = fopen(name,'w');
  fprintf(fileID,'#Volume (Ang^3)   Helmholtz Free Energy (kJ/mol)\n');
  for i = vol_array
    fprintf(fileID,'%f %f\n',i,spline_fvib_func(i));
  end
  fclose(fileID);

  fprintf(results,"%f   %f   %f   %f   %f   %f     %f\n",temp,Vmin,
Fmin,Smin,Hmin,CVmin,F(Energy_fit,Vmin));

end

fclose(results);

disp('Made it to the end')
\end{verbatim}

\subsection{Calculating the Gibbs Free Energy Curve Including Pressure ($G(T,P)$)}

This file just adds one additional for loop to go over the Pressures (in units of GPa).

\begin{verbatim}
%Script to compute Free Energy at given Temp and Volumes
%Self-generates Grueneisen parameters and Fvib calcs
clear

test = fopen('resorcinolAlpha.freq');
ref_found = false;
plus_found = false;
minus_found = false;
ref_freqs = {};
plus_freqs = {};
minus_freqs = {};
x_old = 0;

%constants needed for the calculation
h= 6.62607363e-34; %Js
Na= 6.0221367e23; %1/mol
kb= 1.3806488e-23; %J/K
R= kb*Na;  %J/(mol K)
c= 2.99792458e8; %m/s
count_structures = 0;
%
%Step 0: Extract frequency data from file
%
tline = fgetl(test);
while ~feof(test)
    %disp(tline);
    if contains(tline,'Frequencies')
        blah = strsplit(tline);
        ref_vol = str2double(blah(2));
        ref_found = true;
        tline = fgetl(test);
        x_old = 0;
        count_structures = count_structures + 1;
    elseif contains(tline,'PlusVolume')
        blah = strsplit(tline);
        plus_vol = str2double(blah(2));
        plus_found = true;
        tline = fgetl(test);
        x_old = 0;
        count_structures = count_structures + 1;
    elseif contains(tline,'MinusVolume')
        blah = strsplit(tline);
        minus_vol = str2double(blah(2));
        minus_found = true;
        tline = fgetl(test);
        x_old = 0;
        count_structures = count_structures + 1;
    end
    
    x = str2double(tline);
    
    if (ref_found) && (plus_found) && (minus_found)
        minus_freqs = [minus_freqs x];
    elseif (ref_found) && (plus_found) && (~minus_found)
        plus_freqs = [plus_freqs x];
    elseif (ref_found) && (~plus_found) && (~minus_found)
        ref_freqs = [ref_freqs x];        
    end
    
    if(abs(x_old-x) > 2000) && (x_old~=0)
        %fprintf('x: %f\tx_old: %f\n',x,x_old);
        count_structures = count_structures + 1;
    end
    
    if(x ~= 0)
        x_old = x;
    end
    
    tline = fgetl(test);
end

count_structures = count_structures / 3; 
fprintf('count_structures: %i\n',count_structures);

%convert from a cell array to a regular array
ref_freqs = cell2mat(ref_freqs);
plus_freqs = cell2mat(plus_freqs);
minus_freqs = cell2mat(minus_freqs);

%get rid of NaN in the array
ref_freqs(isnan(ref_freqs)) = [];
plus_freqs(isnan(plus_freqs)) = [];
minus_freqs(isnan(minus_freqs)) = [];
fclose(test);

%
%Step 1: Get electronic energy data and fit it.
%
e_el = importdata('e-el-dft-alpha.dat');
Volumes = e_el.data(:,1)';
Energies = e_el.data(:,2)';
[minEnergy,pos] = min(Energies);
minVolume = Volumes(pos);

F = @(a,x) a(1) + (a(2) * x)/a(3) .* (((a(4)./x).^(a(3)))/(a(3) - 1) + 1 )
 - ((a(2) * a(4) )/(a(3) - 1));

% Set initial guess parameters and perform least squares fit on the E(V)
% data.
a0 = [minEnergy, 5.0, 8.0, minVolume];
Energy_fit = lsqcurvefit(F,a0,Volumes,Energies);
fprintf('E(V) Fit Parameters:\n');
fprintf('   A0 = %.5f\n',Energy_fit(1));
fprintf('   V0 = %.3f\n',Energy_fit(4));
fprintf('   B0 = %.3f\n',Energy_fit(2));
fprintf('   B0prime = %.3f\n\n',Energy_fit(3));
Emin = Energy_fit(1);

name = sprintf( 'murnaghanEqFit.txt');
fitEq = fopen(name,'w');
fprintf(fitEq,'E(V) Fit Parameters:\n');
fprintf(fitEq,'   A0 = %.5f\n',Energy_fit(1));
fprintf(fitEq,'   V0 = %.3f\n',Energy_fit(4));
fprintf(fitEq,'   B0 = %.3f\n',Energy_fit(2));
fprintf(fitEq,'   B0prime = %.3f\n\n',Energy_fit(3));
fclose(fitEq);

%
%Step 2: Begin generating Fvib data
%

%First calculate the grueneisen parameters
grun = -1.*((log(plus_freqs./minus_freqs))./(log(plus_vol./minus_vol)));
grun(isnan(grun)) = 0;

%Volume to extrapolate from
v_min = Energy_fit(4)-100;
v_max = Energy_fit(4)+100;
vol_array = [v_min:v_max];

p0=0:0.01:1.0;
p1=1.5:0.5:5;
p2=6:1:20;
Press=cat(2,p0,p1);
Press=cat(2,Press,p2);

%Pressure in GPa
for press = Press;
  holdName = sprintf('final_resultsP%1.2f.txt', press);
  results = fopen(holdName,'w');
  fprintf(results,'#Temperature (K)     Optimal Volume (Ang^3)   
 Free Energy (kJ/mol)    Entropy (kJ/mol)    Enthalpy (kJ/mol)    
Heat Capacity (J/(mol K))    Internal Energy (kJ/mol)\n');
  
  t0=0:10:100;
  t1=125:25:425;
  Temp=cat(2,t0,t1);
  %Loop over a set of temperatures
  for temp = Temp;
    %Reset Fvib values to an empty array
    fprintf('\n\nTemp = %f\n',temp);
    fvib = {};
    hvib = {};
    svib = {};
    cv = {};
    %loop over a set of volumes
    for vol = vol_array

      %calculate new frequency
      %units: m/s * (cm/m) * (1/cm) = 1/s = Hz
      w =  ( c * 100 ) .* ref_freqs .* (( vol / ref_vol ).^( -1 .* grun ));
      %Calculate zero point energy contribution
      zpe = (sum(w) * h) / 2;
    
      %If relevant calculate other term
      if temp ~= 0
       %units: J/K * K = J
       h_vib_term2 = (h .* w) ./ ( exp( ( h .* w )./( kb * temp )) - 1);
       h_vib_term2 (~isfinite(h_vib_term2))=0;
       h_vib_term2 = sum(h_vib_term2);
       s_vib = ((h .* w) ./(temp .* ( exp( ( h .* w )./( kb * temp )) - 1)))
 - (( kb ) .* log( 1 - exp( ( -1 * h .* w )./( kb * temp ))));
       s_vib(~isfinite(s_vib))=0;
       s_vib= sum(s_vib);   
       Cv = (Na/(count_structures*kb)).*( ( ((h .* w) ./(temp .* 
( exp( ( h .* w )./( kb * temp )) - 1))).^2).* exp( (h .* w )./( kb * temp )));
       Cv(~isfinite(Cv))=0;
       Cv = sum(Cv);     
       %fprintf('entr_enth: %f kJ/mol\n',entropy_enthalpy);       
      else
        entropy_enthalpy = 0.0;
        s_vib = 0;
        h_vib_term2 = 0;
        Cv = 0;
      end
    
      %Sum to create Fvib
      %units: (/mol) * (J + J) * (kJ / J) = kJ/mol
      s_vib =  Na * (s_vib) / (1000 * count_structures);
      h_vib =  Na * (zpe + h_vib_term2) / (1000 * count_structures);
      f_vib = h_vib - (temp * s_vib);
      fvib = [fvib f_vib];
      hvib = [hvib h_vib];   
      svib = [svib s_vib];
      cv = [cv Cv];
    end
    
    %Call to calculate the gibbs free energy
    fvib = cell2mat(fvib);
    hvib = cell2mat(hvib);
    svib = cell2mat(svib);
    cv = cell2mat(cv);
  
    fvib_spline = spline(vol_array,fvib);
    spline_fvib_func = @(v) ppval(fvib_spline,v);
  
    hvib_spline = spline(vol_array,hvib);
    spline_hvib_func = @(v) ppval(hvib_spline,v);  
  
    svib_spline = spline(vol_array,svib);
    spline_svib_func = @(v) ppval(svib_spline,v); 
  
    cv_spline = spline(vol_array,cv);
    spline_cv_func = @(v) ppval(cv_spline,v); 

    PVterm = @(v) press * v * (1.0e-24) * Na; 

    Free_energy = @(v) F(Energy_fit,v) + spline_fvib_func(v) + PVterm(v); 
    
    %Now search over all available volumes for the optimal Gibbs' value
    [Vmin, Fmin, info, output] = fminbnd(Free_energy, v_min, v_max);
  
    % Print out the optimal volume and free energy
    fprintf('Optimal Volume and Free Energy:\n%f Ang^3\t%f kJ/mol\n',
Vmin,Fmin);
    Smin = spline_svib_func(Vmin);
    Hmin = spline_hvib_func(Vmin);
    CVmin = spline_cv_func(Vmin);
    fprintf('Optimal Entropy, Enthalpy and Cv:\n%f J/mol\t%f kJ/mol\t%f
 J/(mol K)\n',Smin*1000,Hmin,CVmin);
    fprintf('PVterm :\t%f kJ/mol\n',PVterm(Vmin));
    
    % Finally add the F(V) result to the plot.
    %plot(Volumes,Free_energy(Volumes) - Fmin);
    %scatter(vol_array,fvib); 
    %hold on; % keep plot active while we add curves
    %plot(vol_array,f_vib_func(vol_array));
    %plot(vol_array,spline_fvib_func(vol_array));
    %legend('Fvib data', 'Fvib lin fit', 'Fvib spline');
    %legend('E(V) data', 'Murnaghan E(V)', 'Free Energy');
    %hold off; % done with plot
  
    %
    %Step 5: Store results in 'logical' files
    %
  
    name = sprintf( 'freeEnergyT%iP%1.2f.txt', temp, press );
    fileID = fopen(name,'w');
    fprintf(fileID,'#Volume (Ang^3)   Free Energy (kJ/mol)\n');
    for i = vol_array
      fprintf(fileID,'%f %f\n',i,Free_energy(i));
      %fprintf('%f %f\n',i,Free_energy(i));
    end

    fclose(fileID);
    
    name = sprintf( 'fvibEnergyT%iP%1.2f.txt', temp, press );
    fileID = fopen(name,'w');
    fprintf(fileID,'#Volume (Ang^3)   Helmholtz Free Energy (kJ/mol)\n');
    for i = vol_array
      fprintf(fileID,'%f %f\n',i,spline_fvib_func(i));
    end
    fclose(fileID);
    
    fprintf(results,"%f   %f   %f   %f   %f   %f     %f
\n",temp,Vmin,Fmin,Smin,Hmin,CVmin,F(Energy_fit,Vmin));

  end
  
  fclose(results);
end

\end{verbatim}

%%% end of main manuscript
\end{spacing}

\end{document}


