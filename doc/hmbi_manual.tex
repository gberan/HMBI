\documentclass[11pt,letterpaper]{article}
\usepackage{epsfig}
\usepackage{setspace}
%\usepackage{fancyvrb} % for making pretty verbatim figures
% see example here: 
% http://x.squidpower.com/2006/08/26/verbatim-code-in-a-framed-figure-box-using-latex/

% Put a header on all pages after the first
\pagestyle{myheadings}
\markright{HMBI User Manual}

% set the margins
\usepackage[letterpaper,margin=1.0in,nofoot,dvips]{geometry}

\makeatletter
\makeatother

\begin{document}

\begin{spacing}{1.0}

% don't put a header on the first page
\thispagestyle{empty}


\section*{\centering {Hybrid Many-Body Interaction QM/MM Code User Manual}}

% don't put a header on the first page
\thispagestyle{empty}

\noindent
\begin{center}
Greg Beran, Kaushik Nanda, and Ali Sebetci.  \\
Department of Chemistry, University of California, Riverside.\\
Last modified: September 2009
\end{center}

%\hline
\tableofcontents
\vspace{5mm}
%\hline

\section{Introduction}

HMBI is great!  Write a real introduction here.




\section{Compilation}

\subsection{Compilation instructions}

First, obtain the source code from CVS.  
\begin{verbatim}
  % cvs co hmbi
\end{verbatim}

This will create a directory 'hmbi' at your current path.  This directory
is refered to as \$HMBI for the remainder of these instructions.
HMBI has serial and parallel versions.  The parallel version uses
MPI, and the execution scripts are designed to use LAM/MPI.  

To compile:
\begin{itemize}
\item Change to the \$HMBI/src directory.

\item To make the serial version ("hmbi.serial"), type:
\begin{verbatim}
  % make serial
\end{verbatim}
(or simply type {\tt make}).  
\item To make both the serial and parallel versions ("hmbi.parallel"), type:
\begin{verbatim}
  % make all
\end{verbatim}
\item To make everything and install it:
\begin{verbatim}
  % make install 
\end{verbatim}
Installing copies the binaries to the \$HMBI/bin directory and creates
a useful softlink in that directory.  
\end{itemize}

To remove all compiled files for a fresh compilation, change to the \$HMBI/src
directory and type:
\begin{verbatim}
  % make clean
\end{verbatim}

Be sure to add the directory \$HMBI/bin to your \$PATH to enable you to run it.

Finally, you should modify \$HMBI/bin/hmbi.run to set the EXE\_PATH, MPIHOME, 
and LAMHOME variables.  The script is set up to use Sun Grid Engine for
parallel jobs.  Alternatively, a "hosts.txt" file listing the name of each
host can be used.  This file can be modified for other batch systems like
PBS.  


\subsection{Software structure}

The structure of the code is as follows:

\vspace{3mm}

\begin{tabular}{ll}
main.C		&- the overall driver for the program \\
params.C	&- a class to store the HMBI job parameters \\
\\
cluster.C	&- a class defining the full cluster, handles most of the real work. \\
dimer.C		&- a class defining a dimer, which consists of 2 monomers \\
monomer.C	&- a class defining a monomer, which consists of atoms \\
atom.C		&- a class defining an atom type \\
multipole.C     &- a class to store multipole expansions. \\
polarizability.C &- a class to store polarizabilities \\
\\
dlf\_interface.C &- interface to the DL-FIND geometry optimization package \\
opt.C		&- a class handling geometry optimization (currently deactived, use DL-FIND)\\
\\
vector.C	&- a vector class \\
matrix.C        &- a matrix class \\
\end{tabular}

\section{Running HMBI}

\subsection{Serial version}

First, prepare your input file (see further instructions below for information
on input file format).  Given an input file 'job.in' and a desired output file
'job.out', run the job by typing:
\begin{verbatim}
 % hmbi job.in > job.out
\end{verbatim}

\subsection{Parallel version}

After creating the input file, create a script file to submit the job
to your cluster scheduler.  On our cluster (which uses the Sun Grid
Engine), the script looks something like:
\begin{verbatim}
-------------------------------------------
#!/bin/sh

#$ -cwd
#$ -pe lammpi 24

hmbi ice16.in > ice16.out
-------------------------------------------
\end{verbatim}

This will run the job using the SGE parallel environment "lammpi" using
24 processors.  You should contact your system administrator to identify
the appropriate parallel environment on your own cluster.
The \$HMBI/bin/hmbi.run script will collect the information about how many
processors to use and pass it to the HMBI code.  


\section{Input format}

The HMBI input consists of a series of sections, which can be found in 
any order. A '\$$<$keyword$>$' marks the start of each section, and '\$end' 
marks the end of each.  The necessary sections are:

\begin{enumerate}
\item {\bf \$comment :} Job title information (optional)

\item {\bf \$molecule :} The molecule specification

\item {\bf \$hmbi :} HMBI keywords

\item {\bf \$qchem :} Q-Chem keywords

\item Either {\bf \$tinker} (Tinker keywords) or {\bf \$orient}
  (Orient keywords) or {\bf \$qchem2} (secondary Q-Chem keywords)
\end{enumerate}

\noindent {\em Optional} sections include:

\begin{enumerate}

\item[6.] {\bf \$unit\_cell :} Periodic boundary condition specification

\item[7.] {\bf \$embedding\_charges :} Specification of embedding charges
  (only works with Tinker MM, not Orient.

\end{enumerate}

\noindent A brief description of each section follows.

\subsection{Job Title/Comments: \$comment} 

	Free format; it can include any comments desired by the user.

\subsection{Molecule Specification: \$molecule}

The format for cluster specification depends on the many-body
treatment type, because some MM types require atom connetivity
information.  The molecular input specification requires only the
minimum information required to define the system for each
type of job.  {\bf Atomic positions are always in units of Angstroms.}

Regardless of the job type, the first line of the \$molecule section
contains the overall charge and spin state of the cluster.
\begin{verbatim}
        <charge> <spin>
\end{verbatim}
The beginning of each monomer is then indicated by a ``{\tt--}'' line,
followed by the monomer specification.  For example, a neutral cluster
with an overall singlet spin state would look like:

\begin{verbatim}
        $molecule
        0 1
        --
        <monomer 1>
        --
        <monomer 2>
        --
        etc.
        $end
\end{verbatim}
\noindent The details of specifying monomers for each particular job
type are discussed below.  

{\bf Note on spin states:} As discussed below, the electronic spin
state for each monomer must be specified.  The algorithms for
determining the spin state of dimers formed from these monomers are
somewhat crude.  Two singlet monomers form a singlet dimer.  Likewise,
a singlet monomer and a doublet monomer combine to form a doublet.
But if two doublets combine, for example, one can obtain either a
singlet or a triplet.  In such cases, the code assigns it to either an
overall singlet (if there is an even number of electrons) or doublet
(if there is an odd number of electrons).

 
\subsubsection{Monomer specification for the {\em ab initio} Force Field (AIFF) many-body model}

For AIFF many-body treatments, the first line of each monomer
specification contains:
\begin{verbatim}
        <charge> <spin>  <ionization potential in a.u.>
\end{verbatim}
The ionization potential (in hartrees) is used in the asymptotic DFT
correction to improve the polarizability predictions.  Experimental or
theoretical ionization potentials can be used.  The results are
moderately sensitive to the particular value of the potential.

The next lines contain the monomer geometry in standard Cartesian XYZ
format.  Each line corresponds to the position of one atom:
\begin{verbatim}
     <atomic symbol> <x> <y> <z>
\end{verbatim}
A sample AIFF-type geometry specification for the (H$_2$O)$_3$ is given below.
\vspace{5mm}
\hrulefill
\begin{verbatim}
        $comment
          Sample AIFF-type geometry specification for water trimer.
        $end

        $molecule
        0 1
        --
        0 1 0.4638
        O     -1.201363      0.936340      0.045523
        H     -1.875414      1.146057     -0.611433
        H     -1.109807     -0.037296      0.018825
        --
        0 1 0.4638
        O     -0.044745     -1.610729     -0.008393 
        H      0.043174     -2.205957      0.745669 
        H      0.754975     -1.048497      0.017871 
        --
        0 1 0.4638
        O      1.585266      0.661860      0.055793 
        H      2.142455      1.072502     -0.615606 
        H      0.711544      1.090261     -0.036259 
        $end
\end{verbatim}
\hrulefill
\vspace{5mm}

Note: for the AIFF generation, a set of local coordinates will be
defined for each monomer.  These are defined as: The first atom of
each monomer is the origin of the local coordinate system of that
monomer. The positive z-axis of the local coordinate system is defined
as the axis from the first atom to the second atom. xz-plane of the
local coordinate system is the plane of the first three atoms of the
monomer. For diatomic monomers, a dummy third atom may be added.



\subsubsection{Monomer specification for Quantum or EE-PA many-body models}
Jobs in which a lower-level quantum mechanical method (such as Hartree-Fock)
or the electrostatically embedded pairwise-additive approximation is used
use a molecular specification utilize a nearly identical molecular specification
to the AIFF many-body jobs.  The only difference is that the ionization
potential is omitted on the charge/spin line.  For example, one water
monomer would be specified as:
\vspace{5mm}
\hrulefill
\begin{verbatim}
        --
        0 1
        O     -1.201363      0.936340      0.045523
        H     -1.875414      1.146057     -0.611433
        H     -1.109807     -0.037296      0.018825
\end{verbatim}
\hrulefill



\subsubsection{Monomer specification for {\tt Tinker} force fields (such as Amoeba)}
The {\tt Tinker} software package requires atomic connectivity
information in addition to the atomic coordinates.  The first line of the
specification defines the charge and spin.  Subesequent lines define the
atoms as:
\begin{verbatim}
     <counter> <atomic symbol> <x> <y> <z> <atom_type>
 <connectivity>
\end{verbatim}
The {\tt counter} is a simple counter for the atoms whose number
starts at 1 on each monomer.  Fragment boundaries should not disrupt
chemical bonds.  As always, the {\tt xyz} Cartesian coordinates are in
Angstroms.  The {\tt atom\_type} is the molecular mechanics atom type,
and it corresponds to the atom type in the chosen force field.  Force-field
specification is described in Section~\ref{sec-tinker}.
and {\tt connectivity} is a list of up to 6 integers denoting the
other atoms to which this atom is bonded.  A sample specification for
the water trimer is shown below.

\vspace{5mm}
\hrulefill
\begin{verbatim}
        $comment
          Sample Tinker-type geometry specification for water trimer.
          The atom number corresponds to the Amoeba force field.
        $end

        $molecule
        0 1
        --
        0 1 0.4638
        1  O     -1.201363      0.936340      0.045523  22  2  3
        2  H     -1.875414      1.146057     -0.611433  23  1
        3  H     -1.109807     -0.037296      0.018825  23  1
        --
        0 1 0.4638
        1  O     -0.044745     -1.610729     -0.008393  22  2  3 
        2  H      0.043174     -2.205957      0.745669  23  1
        3  H      0.754975     -1.048497      0.017871  23  1 
        --
        0 1 0.4638
        1  O      1.585266      0.661860      0.055793  22  2  3
        2  H      2.142455      1.072502     -0.615606  23  1
        3  H      0.711544      1.090261     -0.036259  23  1
        $end
\end{verbatim}
\hrulefill
\vspace{5mm}


\subsection{HMBI Keyword Specification: \$hmbi}
A series of keywords of the format ``{\tt <parameter> = <value>}''.
See Chapter~\ref{HMBI_keywords} for more details on the specific
keywords.

\subsection{Q-Chem Keyword Specification: \$qchem and/or \$qchem2} 
	A series of Q-Chem keywords in standard Q-Chem format. This
	section becomes the \$rem section in Q-Chem inputs.   Caution,
        not all keywords work well.  For example, frozen core orbital
        keywords are messy, since the number of frozen core orbitals
        depends on whether we are looking at a monomer or dimer.

\subsection{Tinker Keyword Specification: \$tinker \label{sec-tinker}}
   	A series of Tinker keywords in standard Tinker format.  This
	section becomes the *.key file for Tinker.  If this section is
        present, there should not be a \$aiff section.

\subsection{AIFF Keyword Specification: \$aiff}
   	A series of keywords to be used in the CamCasp/Dalton and Orient
        program packages.  If this section is
        present, there should not be a \$tinker section.

\vspace{3mm}
\noindent
Keyword: CamCaspHome \\
Values:  The home directory of the CamCasp program package. \\
Default: None \\
Example: /home/software/camcasp-5.2.00   \\

\vspace{3mm}
\noindent
Keyword: OrientBasisSet \\
Values:  Sadlej, cc-pV$X$Z, or aug-cc-pV$X$Z (where $X$ = D, T, or Q). \\
Default: Sadlej \\
Notes: This is the basis set used to compute the AIFF multipole moments
and polarizabilities. \\

\vspace{3mm}
\noindent
Keyword: DampingFactor \\
Values: Floating point number \\
Default: None \\
Notes: This prevents the ``polarization catastrophe'' at short ranges
when computing self-consistent induction energies.  Typical values
range roughly 1.5-2.0. For water, 1.45 appears to work well.  This
parameter should be determined empirically by benchmarking against a
set of fully quantum mechanical calculations. \\

\noindent
A sample AIFF input variable section:
\begin{verbatim}
        $aiff
        CamCaspHome = /home/software/camcasp-5.2.00
        OrientBasisSet = sadlej
        DampingFactor = 1.45
        $end
\end{verbatim}

\subsection{Periodic Boundary Conditions Specification: \$unit\_cell}
	There are two options for specifying the unit cell.

	The default uses the lengths of each side of the unit cell (a,
        b, \& c) and the angles between them (alpha, beta, and gamma),
        where alpha is the angle between axes b \& c, beta is the
        angle between a \& c, and gamma is the angle between a \& b.
        Axis lengths are in Angstroms, and angles in degrees.  For
        example, here is the specification for formamide crystal,
        which has (a,b,c) = (3.5432, 8.9512, 6.9741), alpha = gamma =
        90 degrees, and beta = 101.051 degrees.

\begin{verbatim}
	$unit_cell
	3.5432 8.9512 6.9741
	90.0 101.051 90.0
	$end
\end{verbatim}
	The other format, which is activated by setting the \$hmbi
        keyword {\tt READ\_LATTICE\_VECTORS = TRUE}, looks for a list
        of three lattice vectors (as row vectors) that define the unit
        cell.  For example, if the three vectors are (7.569, 0, 0),
        (0, 5.366, 0), and (-4.745, 0, 8.552), you would input:

\begin{verbatim}
	$unit_cell
	 7.569 0.000 0.000
	 0.000 5.366 0.000
	-4.756 0.000 8.552
	$end
\end{verbatim}


\subsection{\$embedding\_charges}
	Charges are listed, in a.u., by fragment, in the same order as
        in the \$molecule section, with "--" separating each fragment.
        For example, in a system with two water molecules, you might
        have:

\begin{verbatim}
	$embedding_charges
	--
	-0.7781
	0.3891
	0.3891
	--
	-0.7781
	0.3891
	0.3891
	$end
\end{verbatim}



\subsection{Sample Input File}
Here is a sample input file on a water trimer.

\begin{verbatim}
$comment
Water Trimer
$end

$molecule
0 1
--
0 1
1  O       -0.571270    2.322980   -0.003174  22  2  3
2  H       -0.541626    2.994645    0.657473  23  1
3  H        0.319714    1.986524   -0.080413  23  1
--
0 1
1  O        2.038411    1.261129   -0.102281  22  2  3
2  H        2.545757    1.390343   -0.886606  23  1
3  H        2.008500    0.316696    0.040400  23  1
--
0 1
1  O        1.831197   -1.536089    0.156570  22  2  3
2  H        2.127985   -1.971116    0.938635  23  1
3  H        0.922213   -1.803155    0.031130  23  1
$end

$hmbi
jobtype = energy
path_qm = rimp2 
path_mm = amoeba
mm_code = tinker
periodic = false
iprint = 0
local_2_body = true
cutoff1 = 7.1
cutoff0 = 8.1
$end

$qchem
exchange = hf
basis = aug-cc-pvdz
aux_basis = rimp2-aug-cc-pvdz
correlation = rimp2
purecart = 11111
thresh = 14
scf_convergence = 8
mem_static = 200
mem_total = 2000
symmetry = false
$end

$tinker         
# Force Field Selection
PARAMETERS        /home/software/tinker/params/amoeba.prm
# Precision         
DIGITS            8
$end            
\end{verbatim}



\section{HMBI Keywords {\label{HMBI_keywords}}}

The keywords in the \$hmbi section can be entered in any order, and with 
any capitalization or spacing.  The only requirement is that they are 
listed in the format "parameter = value".  An exception is for the 
{\tt QM\_PATH} and {\tt MM\_PATH} keywords, for which the listed paths are case 
sensitive and must adhere to standard Unix/Linux rules.

The section begins with \$hmbi, and ends with \$end.  Blank lines in 
the section are allowed, but comment lines are not. A list of keywords follows:
\\

\noindent
Keyword: {\tt JOBTYPE} \\
Values:  {\tt SP} (or {\tt SINGLEPOINT} or {\tt ENERGY}) = Do an HMBI 
         energy calculation [default] \\
	 {\tt FORCE} = Do an HMBI gradient calculation \\
         {\tt OPT} = Optimize the geometry \\
	 {\tt HESSIAN} (or {\tt FREQ} or {\tt FREQUENCY}) = 
         Do an HMBI frequency calculation.  Not yet implemented.\\
	 {\tt EXPAND} = Take the input geometry, and scale the distance of each
	 molecule from the center of mass.  A tool for developing potential
	 energy surfaces that expand/shrink the cluster intermolecular spacings.
	 Requires the keyword {\tt EXPANSION\_FACTOR}  as well. \\
Notes:   The {\tt EXPAND} jobtype creates two new files, a cartesian xyz file,
	 new\_geom.xyz, that can be visualized with Molden or other software,
	 and a new input file, which is a direct copy of the original except
	 it has the new geometry. \\ \\ 
\noindent
Keyword: QM\_PATH or MM\_PATH \\
Values:  any valid Unix-type path, pointing to the QM or MM jobs 
	 (case-sensitive). \\
Notes:   The path should be local (expressed in terms of the directory from
	 which the program is being executed). \\

\noindent
Keyword: MM\_CODE \\
Values:  TINKER, ORIENT \\
%	 ORIENT (not yet implemented) \\
Notes:   Tells the software which type of MM output to read. \\

\noindent
Keyword: IPRINT \\
Values:  0 or any positive integer \\
Default: 0 \\
Notes:   Higher values print out more output \\

\noindent
Keyword: PERIODIC \\
Values:  True or False  \\
Default: False \\
Notes:   If True, must specify the \$unit\_cell section. \\

\noindent
Keyword: READ\_LATTICE\_VECTORS \\
Values:  True or False \\
Default: False \\
Notes:   If False, the code looks for axis lengths (a,b,c) and angles
         (alpha, beta, gamma) to define unit cell.  If True, the code
         looks for three lattice vectors, v1, v2, \& v3, each of the form
         (xi,yi,zi).  See the manual section regarding the \$unit\_cell
	 section for more details. \\

\noindent
Keyword: COUNTERPOISE \\
Values:  True or False  \\
Default: False \\
Notes:   If True, it invokes Q-Chem's Jobtype = BSSE and extracts Counterpoise
	 corrected energies.  Warning: Q-Chem's BSSE jobtype is not fully
	 compatible with all Q-Chem methods (e.g. dual-basis SCF). \\

\noindent
Keyword: EMBEDDING\_CHARGES \\
Values:  True or False  \\
Default: False \\
Notes:   If True, it uses embedding charges to lessen the importance of 
many-body
	 terms.  The input file section \$embedding\_charges must be used in 
	 conjuction with this keyowrd. \\

\noindent
Keyword: LOCAL\_2\_BODY \\
Values:  True of False \\
Default: False \\
Notes:   Turns on smooth local truncation of pairwise interactions.  Truncated
	 2-body terms are treated at the MM level.  Set in combination with
         CUTOFF1 and CUTOFF0. \\

\noindent
Keyword: CUTOFF1 \\
Values:  Any positive real number \\
Default: False \\
Notes:   Cutoff (in Angstroms) below which all 2-body interactions are treated 
	 at the QM level. \\

\noindent
Keyword: CUTOFF0 \\
Values:  Any positive real number \\
Default: CUTOFF1 + 1.0  \\
Notes:   Cutoff (in Angstroms) above which all 2-body interactions are treated 
	 at the MM level.  Between CUTOFF1 and CUTOFF0, QM 2-body interactions 
	 are smoothly damped. \\
 
\noindent 
Keyword: NEGLECT\_MANY\_BODY \\
Values:  True of False \\
Default: False
Notes:   Turns off MM many-body terms.  Purely for debugging or testing. \\


\noindent
Keyword: EXPANSION\_FACTOR \\
Values:  Any positive real number \\
Default: 1.0 \\
Notes:   Scale factor for expanding or contracting the cluster geometry,
	 in coordination with JOBTYPE = EXPAND.  Values $>$ 1 expand, and 
	 Values $<$ 1 shrink it. \\

\noindent
Keyword: MAX\_OPT\_CYCLES \\
Values:  Positive integer \\
Default: 100 \\
Notes:   Maximum number of cycles to use in geometry optimizations.  
         Default is 100 at present.  





%%% end of main manuscript
\end{spacing}

\end{document}


